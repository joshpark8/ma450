% \section{Euclidean Domains}

\begin{definition}[Euclidean Domain (ED)]
	An integral domain $D$ is called a \textit{Euclidean domain} if there is a function $d$ (called the \textit{measure}) from nonzero elements of $D$ to the nonnegative integers such that
	\begin{enumerate}
		\item $d(a) \leq d(ab)$ for all nonzero $a,b \in D$; and
		\item if $a,b \in D,\ b \neq 0$, then there exist elements $q$ and $r$ in $D$ such that $a = bq + r$, where $r = 0$ or $d(r) < d(b)$.
	\end{enumerate}
\end{definition}

\begin{theorem}[ED Implies PID]
	Every Euclidean domain is a principal ideal domain.
\end{theorem}

\begin{corollary}[ED Implies UFD]
	Every Euclidean domain is a unique factorization domain.
\end{corollary}

\begin{theorem}[$\mathbf{D}$ a UFD Implies $\mathbf{D[x]}$ a UFD]
	If $D$ is a unique factorization domain, then $D[x]$ is a unique factorization domain.
\end{theorem}
