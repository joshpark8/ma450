% \section{Motivation and Definition}

\begin{definition}[Ring]
	A \textit{ring} $R$ is a set with two binary operations, addition (denoted by $a + b$) and multiplication (denoted by $ab$), such that for all $a,b,c$ in $R$:
	\begin{enumerate}
		\item $a + b = b + a$.
		\item $(a + b) + c = a + (b + c)$.
		\item There is an additive identity 0. That is, there is an element 0 in $R$ such that $a + 0 = a$ for all $a$ in $R$.
		\item There is an element $-a$ in $R$ such that $a + (-a) = 0$.
		\item $a(bc) = (ab)c$.
		\item $a(b+c) = ab + ac$ and $(b + c)a = ba + ca$.
	\end{enumerate}
\end{definition}

% \begin{remark}
% 	Note that multiplication need not be commutative. When it is, we say that the ring is \textit{commutative}. Also, a ring need not have an identity under multiplication. A \textit{unity} (or \textit{identity}) in a ring is a nonzero element that is an identity under multiplication. A nonzero element of a com-
% 	mutative ring with unity need not have a multiplicative inverse. When it does, we say that it is a unit of the ring. Thus, $a$ is a unit if $a^{-1}$ exists.

% 	\noindent The following terminology and notation are convenient. If $a$ and $b$ belong to a commutative ring $R$ and $a$ is nonzero, we say that $a$ \textit{divides} $b$ (or that $a$ is a \textit{factor} of $b$) and write $a \vert b$, if there exists an element $c$ in $R$ such that $b = ac$. If $a$ does not divide $b$, we write $a \nmid b$.
% \end{remark}
