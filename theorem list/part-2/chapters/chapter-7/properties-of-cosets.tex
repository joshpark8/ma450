\section{Properties of Cosets}

\begin{definition}[Coset of $\mathbf{H}$ in $\mathbf{G}$]
	Let $G$ be a group and let $H$ be a nonempty subset of $G$. For any $a \in G$, the set $\{ah\ \vert\ h \in H\}$ is denoted by $aH$. Analogously, $Ha = \{ha\ \vert\ h \in H\}$ and $aHa^{-1} = \{aha^{-1}\ \vert\ h \in H\}$. When $H$ is a subgroup of $G$, the set $aH$ is called the \textit{left coset of $H$ in $G$ containing $a$}, whereas $Ha$ is called the \textit{right coset of $H$ in $G$ containing $a$}. In this case, the element $a$ is called the \textit{coset representative of $aH$ (or $Ha$)}. We use $\abs{aH}$ to denote the number of elements in the set $aH$, and $\abs{Ha}$ to denote the number of elements in $Ha$.
\end{definition}

\begin{lem}[ Properties of Cosets]
	Let $H$ be a subgroup of $G$, and let $a$ and $b$ belong to $G$. Then,
	\begin{enumerate}
		\item $a \in aH$.
		\item $aH = H$ if and only if $a \in H$.
		\item $(ab)H = a(bH)$ and $H(ab) = (Ha)b$.
		\item $aH = bH$ if and only if $a \in bH$.
		\item $aH = bH$ or $aH \cap bH = \emptyset$.
		\item $aH = bH$ if and only if $a^{-1}b \in H$.
		\item $\abs{aH}=\abs{bH}$.
		\item $aH = Ha$ if and only if $H = aHa^{-1}$.
		\item $aH$ is a subgroup of $G$ if and only if $a \in H$.
	\end{enumerate}
\end{lem}
