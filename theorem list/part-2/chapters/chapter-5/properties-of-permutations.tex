\section{Properties of Permutations}

\begin{theorem}[Products of Disjoint Cycles]
	Every permutation of a finite set can be written as a cycle or as a product of disjoint cycles.
\end{theorem}

\begin{theorem}[Disjoint Cycles Commute]
	If the pair of cycles $\alpha = (a_1, a_2, \dots, a_m)$ and $\beta = (b_1, b_2, \dots, b_n)$ have no entries in common, then $\alpha\beta = \beta\alpha$.
\end{theorem}

\begin{theorem}[Order of a Permutation]
	The order of a permutation of a finite set written in disjoint cycle form is the least common multiple of the lengths of the cycles.
\end{theorem}

\begin{theorem}[Product of 2-Cycles]
	Every permutation in $S_n,\ n>1$ is a product of 2-cycles.
\end{theorem}

\begin{lem}
	If $\varepsilon = \beta_1\beta_2\dots\beta_r$, where the $\beta$'s are 2-cycles, then $r$ is even.
\end{lem}

\begin{theorem}[Always Even or Always Odd]
	If a permutation $\alpha$ can be expressed as a product of an even (odd) number of 2-cycles, then every decomposition of $\alpha$ into a product of 2-cycles must have an even (odd) number of 2-cycles. In symbols, if
	\[ \alpha = \beta_1\beta_2\dots\beta_r\ \ \ \ \text{and}\ \ \ \ \alpha=\gamma_1\gamma_2\dots\gamma_s, \]
	where the $\beta$'s and the $\gamma$'s are 2-cycles, then $r$ and $s$ are both even or both odd.
\end{theorem}

\begin{definition}[Even and Odd Permutations]
	A permutation that can be expressed as a product of an even number of 2-cycles is called an \textit{even} permutation. A permutation that can be expressed as a product of an odd number of 2-cycles is called an \textit{odd} permutation.
\end{definition}

\begin{theorem}[Even Permutations Form a Group]
	The set of even permutations in $S_n$ forms a subgroup of $S_n$.
\end{theorem}

\begin{definition}[Alternating Group of Degree $\mathbf{n}$]
	The group of even permutations of $n$ symbols is denoted by $A_n$ and is called the \textit{alternating group of degree $n$}.
\end{definition}

\begin{theorem}
	For $n > 1$, $A_n$ has order $n!/2$.
\end{theorem}
