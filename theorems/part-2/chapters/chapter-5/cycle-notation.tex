\section{Cycle Notation}
\begin{definition}
	Consider the permutation
	\[ \alpha = \begin{bmatrix}
			1 & 2 & 3 & 4 & 5 & 6 \\
			2 & 1 & 4 & 6 & 5 & 3
		\end{bmatrix}\]

	The assignment of values is as follows:
	\begin{align*}
		1 \mapsto 2 \qquad
        2 \mapsto 1 \qquad
        3 \mapsto 4 \qquad
        4 \mapsto 6 \qquad
        5 \mapsto 5 \qquad
        6 \mapsto 3
	\end{align*}

	Although mathematically satisfactory, such diagrams are cumbersome. Instead, we leave out the arrows and simply write $\alpha = (1,2)(3,4,6)(5)$.

	It is also worth noting that an expression of the form $(a_1, a_2, \dots, a_m)$ is called a \textit{cycle of length $m$}, or an \textit{$m$-cycle}.
\end{definition}
\begin{example}
	To multiply cycles, consider the following permutations from $S_8$. Let $\alpha = (13)(27)(456)(8)$ and $\beta = (1237)(648)(5)$. (When the domain consists of single-digit integers, it is common practice to omit the commas between the digits.) What is the cycle form of $\alpha\beta$? Of course, one could say that $\alpha\beta = (13)(27)(456)(8)(1237)(648)(5)$, but it is usually more desirable to express a permutation in a \textit{disjoint} cycle form (that is, the various cycles have no number in common). Well, keeping in mind that function composition is done from right to left and that each cycle that does not contain a symbol fixes the symbol, we observe that $(5)$ fixes 1; $(648)$ fixes $1$; $(1237)$ sends 1 to 2, $(8)$ fixes 2; $(456)$ fixes 2; $(27)$ sends 2 to 7; and $(13)$ fixes 7. So the net effect of $\alpha\beta$ is to send 1 to 7. Thus, we begin $\alpha\beta=(17\dots)\dots$. Now, repeating the entire process beginning with 7, we have, cycle by cycle, right to left,
	\[ 7 \to 7 \to 7 \to 1 \to 1 \to 1 \to 1 \to 3, \]
	so that $\alpha\beta = (173\dots)\dots$. Ultimately, we have $\alpha\beta = (1732)(48)(56)$. The important thing to bear in mind when multiplying cycles is to "keep moving" from one cycle to the next from right to left.
\end{example}
