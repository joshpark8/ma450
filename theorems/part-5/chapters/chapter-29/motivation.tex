\section{Motivation}

\begin{remark}
	In general, we say that two designs (arrangements of beads) $A$ and $B$ are \textit{equivalent under a group $G$} of permutations of the arrangements if there is an element $\phi$ in $G$ such that $\phi(A) = B$. That is, two designs are equivalent under $G$ if they are in the same orbit of $G$. It follows, then, that the number of nonequivalent designs under $G$ is simply the number of orbits of designs under $G$. (The set being permuted is the set of all possible designs or arrangements.)
\end{remark}
