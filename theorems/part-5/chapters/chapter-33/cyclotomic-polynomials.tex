\section{Cyclotomic Polynomials}

\begin{remark}
	Recall from Example 2 in Chapter 16 that the complex zeros of $x^n-1$ are 1, $\omega = \cos(2\pi/n) = i\sin(2\pi/n)$, $\omega^2, \omega^3,\dots,\omega^{n-1}$. Thus, the splitting field of $x^n-1$ over $\Q$ is $\Q(\omega)$. This field is called the \textit{$n$th cyclotomic extension of $\Q$}, and the irreducible factors of $x^n-1$ over $\Q$ are called the \textit{cyclotomic polynomials}.

	Since $\omega=\cos(2\pi/n) + i\sin(2\pi/n)$ generates a cyclic group of order $n$ under multiplication, we know from Corollary 3 of Theorem 4.2 that the generators of $\cyc{\omega}$ are the elements of the form $\omega^k$, where $1 \leq k \leq n$ and $\gcd(n,k) = 1$. These generators are called the \textit{primitive $n$th roots of unity}. Recalling that we use $\phi(n)$ to denote the number of positive integers less than or equal to $n$ and relatively prime to $n$, we see that for each positive integer $n$ there are precisely $\phi(n)$ primitive $n$th roots of unity. The polynomials whose zeros are the $\phi(n)$ primitive $n$th roots of unity have a special name.
\end{remark}

\begin{definition}
	For any positive integer $n$, let $\omega_1,\omega_2,\dots,\omega_{\phi(n)}$ denote the primitive $n$th roots of unity. the \textit{$n$th cyclotomic polynomial over $\Q$} is the polynomial $\Phi_n(x) = (x-\omega_1)(x-\omega_2)\dots(x-\omega_{\phi(n)})$.
\end{definition}

\begin{theorem}
	For every positive integer $n$, $x^n-1 = \Pi_{d\vert n}\Phi_d(x)$, where the product runs over all positive divisors $d$ of $n$.
\end{theorem}

\begin{theorem}
	For every positive integer $n$, $\Phi_n(x)$ has integer coefficients.
\end{theorem}

\begin{theorem}[(Gauss)]
	The cyclotomic polynomials $\Phi_n(s)$ are irreducible over $\Z$.
\end{theorem}

\begin{theorem}
	Let $\omega$ be a primitive $n$th root of unity. Then $\gal(\Q(\omega)/\Q) \approx U(n)$.
\end{theorem}
