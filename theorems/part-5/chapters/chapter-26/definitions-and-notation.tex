\section{Definitions and Notation}

\begin{remark}
	For any set $S=\{a,b,c,\dots\}$ of distinct symbols, we create a new set $S^{-1} = \{a^{-1},b^{-1},c^{-1},\dots\}$ by replacing each $x$ in $S$ by $x^{-1}$. Define the set $W(S)$ to be the collection of all formal finite strings of the form $x_1x_2\dots x_k$, where each $x_i \in S \cup S^{-1}$. The elements of $W(S)$ are called \textit{words from $S$}. We also permit the string with no elements to be in $W(S)$. this word is called the \textit{empty word} and is denoted by $e$.

	We may define a binary operation on the set $W(S)$ by juxtaposition; that is, if $x_1x_2\dots x_k$ and $y_1y_2\dots y_t$ belong to $W(S)$, then so does $x_1x_2\dots x_ky_1y_2\dots y_t$. Observe that this operation is associative and the empty word is the identity. Also, notice that a word such as $aa^{-1}$ is not the identity, because we are treating the elements of $W(S)$ as formal symbols with no implied meaning.
\end{remark}

\begin{definition}[Equivalence Classes of Words]
	For any pair of elements $u$ and $v$ of $W(S)$, we say that $u$ is related to $v$ if $v$ can be obtained from $u$ by a finite sequence of insertions or deletions of words of the form $xx^{-1}$ or $x^{-1}x$, where $x \in S$.
\end{definition}
