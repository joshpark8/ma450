\section{Group Action}

\begin{remark}
	Our informal approach to counting the number of objects that are considered nonequivalent can be made formal as follows. If $G$ is a group and $S$ is a set of objects, we say that $G$ \textit{acts on} $S$ if there is a homomorphism $\gamma$ from $G$ to sym$(S)$, the group of all permutations on $S$. (The hommomorphism is sometimes called the \textit{group action}.) For convenience, we denote the image of $g$ under $\gamma$ as $\gamma_g$. Then two objects $x$ and $y$ in $S$ are viewed as equivalent under the action of $G$ if and only if $\gamma_g(x) = y$ for some $g$ in $G$. Notice that when $\gamma$ is one-to-one, the elements of $G$ may be regarded as permutations on $S$. On the other hand, when $\gamma$ is not one-to-one, the elements of $G$ may still be regarded as permutations on $S$, but there are distinct elements $g$ and $h$ in $G$ such that $\gamma_g$ and $\gamma_h$ induce the same permutation on $S$ [that is, $\gamma_g(x) = \gamma_h(x)$ for all $x$ in $S$]. Thus, a group acting on a set is a natural generalization of the permutation group concept.
\end{remark}
