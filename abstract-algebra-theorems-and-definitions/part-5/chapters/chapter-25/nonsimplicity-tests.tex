\section{Nonsimplicity Tests}

\begin{theorem}[Sylow Test for Nonsimplicity]
	Let $n$ be a positive integer that is not prime, and let $p$ be a prime divisor of $n$. If 1 is the only divisor of $n$ that is equal to 1 modulo $p$, then there does not exist a simple group of order $n$.
\end{theorem}

\begin{theorem}[$\mathbf{2\cdot}$Odd Test]
	An integer of the form $2 \cdot n$, where $n$ is an odd number greater than 1, is not the order of a simple group.
\end{theorem}

\begin{theorem}[Generalized Cayley Theorem]
	Let $G$ be a group and let $H$ be a subgroup of $G$. Let $S$ be the group of all permutations of the left cosets of $H$ in $G$. Then there is a homomorphism from $G$ into $S$ whose kernel lies in $H$ and contains every normal subgroup of $G$ that is contained in $H$.
\end{theorem}

\begin{corollary}[Index Theorem]
	If $G$ is a finite group and $H$ is a proper subgroup of $G$ such that $\abs{G}$ does not divide $\abs{G:H}!$, then $H$ contains a nontrivial normal subgroup of $G$. In particular, $G$ is not simple.
\end{corollary}

\begin{corollary}[Embedding Theorem]
	If a finite non-Abelian simple group $G$ has a subgroup of index $n$, then $G$ is isomorphic to a subgroup of $A_n$.
\end{corollary}
