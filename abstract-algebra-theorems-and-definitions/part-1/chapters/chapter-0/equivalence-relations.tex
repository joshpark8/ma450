\section{Equivalence Relations}

\begin{definition}[Equivalence Relation]
	An \textit{equivalence relation} on a set $S$ is a set $R$ of ordered pairs of elements of $S$ such that

	\begin{enumerate}
		\item $(a, a) \in R$ for all $a \in S$  (reflexive property).
		\item $(a, b) \in R$ implies $(b, a) \in R$  (symmetric property).
		\item $(a, b) \in R$ and $(b, c) \in R$ imply $(a, c) \in R$  (transitive property).
	\end{enumerate}
\end{definition}

\begin{definition}[Partition]
	A \textit{partition} of a set $S$ is a collection of nonempty disjoint subsets of $S$ whose union is $S$.
\end{definition}

\begin{theorem}[Equivalence Classes Partition]
	The equivalence classes of an equivalence relation on a set $S$ constitute a partition of $S$. Conversely, for any partition $P$ of $S$, there is an equivalence relation on $S$ whose equivalence classes are the elements of $P$.
\end{theorem}
