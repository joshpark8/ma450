\subsection*{Lecture 28 (10/30)} % wed oct 30

To recap last lecture, the \tun{\textbf{Fundamental Theorem of Finite Abelian Groups}} states:
\[\text{\(G\) finite abelian group \(\implies\) \(\order{G} = p_1^{n_1}p_2^{n_2}\cdots p_k^{n_k}\)}\]

By \tun{\textbf{Lemma 1}}, \(G = G(p_1) \idp G(p_2) \idp \cdots \idp G(p_k)\) where each \(G(p_i)\) has order \(p_i^{n_i}\).

By \tun{\textbf{Lemma 2}}, each \(G(p_i)\) = internal direct product of cyclic groups, each has order of some power of \(p_i\)

\setcounter{section}{23}
\section{Sylow's Theorem}

\begin{definition}[Conjugate class of \(a\)]

    \(a,b\in G\) are called \tun{conjugate} in \(G\) if \(b=xax\inv\) for some \(x\in G\).

    The \tun{conjugate class of \(a\)} is the set \(\cl{a} = \{\conj{x}{a} \st x\in G\}\).
\end{definition}

\begin{remark}
    Conjugacy is an equivalence relation on \(G\).
\end{remark}

\begin{example}
    \(D_4 = \{ R_{0},\  R_{90},\  R_{180},\  R_{270},\  F_{0},\  F_{45},\ F_{90},\ F_{135} \}\)
    \begin{align*}
        \cl{R_{0}} &= \{R_{0}\} \qquad & \cl{R_{90}}&=\{R_{90}, R_{270}\} = \cl{R_{270}} \qquad & \cl{R_{180}} &= \{R_{180}\} \\
            \cl{F_{0}} &= \{F_{0}, F_{90}\} = \cl{R_{90}} \qquad & \cl{F_{45}} &= \{F_{45}, F_{135}\}
    \end{align*}
\end{example}

\begin{theorem}[24.1]
    Let \(G\) be a finite group and \(a\in G\). Then, \(\order{\cl{a}}=\index{G}{C(a)}\).
\end{theorem}

\begin{proof}[Proof of Theorem 24.1]
Recall \(C(a) = \{h\in G \st ha=ah\}\) is the \tun{centralizer of \(a\) in \(G\)} and \(C(a)\sgp G\).

Consider \( \quad\begin{matrix}
    G\to \cl{a} \\
    \ \ x\mapsto \conj{x}{a}
\end{matrix}\quad \) induces a map \(T:\begin{matrix}
    \{\text{left cosets of } C(a)\} \to \cl{a} \\
    \qquad \qquad \qquad \quad xC(a) \mapsto \conj{x}{a}
\end{matrix}\).

\begin{itemize}
    \item \(T\) is well-defined if \begin{align*}
        xC(a) = yC(a) &\iff x=yh \text{ for some } h\in C{a} \\
            &\implies \conj{x}{a} = \conj{y}{\conj{h}{a}} = \conj{y}{a}
    \end{align*}
    \item \(T\) is onto (obvious)
    \item \(T\) is 1-1: \begin{align*}
        \conj{x}{a} = \conj{y}{a} &\implies (y\inv x)a = a(y\inv x) \\
        &\implies y\inv x \in C(a) \\
        &\implies xC(a) = yC(a)
    \end{align*}
\end{itemize}
Since \(T\) is a 1-1 correspondence, we know that \begin{align*}
    \order{\cl{a}} &= \text{\# of left cosets of C(a)} = \index{G}{C(a)} = \frac{\order{G}}{\order{C(a)}}
\end{align*}
\end{proof}

\begin{corollary}
  \(\order{\cl{a}}\divs \order{G}\) for any \(a\in G\)
\end{corollary}

\begin{proof}[Proof of Corollary]
    \(\order{\cl{a}} = \frac{\order{G}}{\order{C(a)}} \Big\vert \order{G}\)
\end{proof}

\begin{corollary}
  For any finite group \(G\), \begin{align*}
    \order{G} = \sum \index{G}{C(a)}
  \end{align*}
  where the sum runs over one element \(a\) from each conjugacy class of \(G\).
\end{corollary}
\begin{proof}[Proof of Corollary]
    \begin{align*}
        \order{G} &= \sum_a \order{\cl{a}}\qquad\text{(sum runs over)} \\
        &= \sum \index{G}{C(a)}
    \end{align*}
\end{proof}

\begin{theorem}
    Let \(G\) be a finite group such that \(\order{G} = p^n\) where \(n\geq 1\). Then \(Z(G)\) has more than one element.
\end{theorem}

\begin{proof}[Proof of Theorem 24.2]
    Notice that \(a\in Z(G) \iff \cl{a}=\{a\}\)

    Thus we have that \begin{align*}
        \order{G} = \order{Z(G)} + \sum\index{G}{C(a)} = \sum \order{\cl{a}}
    \end{align*} where the above sum runs over representatives of all conjugacy classes with more than one element
    \begin{align*}
        \index{G}{C(a)} &= \frac{\order{G}}{\order{C(a)}} = p^k\text{ with } k \geq 1 \\
        \implies \order{Z(G)} &= \order{G} - \sum\index{G}{C(a)} = p^n - \sum p^k \text{ divisible by p} \\
        \implies \order{Z(G)} &\neq 1
    \end{align*}
\end{proof}

\begin{corollary}
  If \(\order{G} = p\sq\) where p prime, then \(G\) abelian.
\end{corollary}

\begin{proof}[Proof of Corollary]
    \(\order{Z(G)} \divs p^2\) and \(\order{Z(G)}\neq 1\) (by Thm) \(\implies \order{Z(G)}=p\) or \(p\sq\)
    \begin{align*}
        \text{If } \order{Z(G)} = p\sq &\implies G = Z(G) \\
        &\implies G \text{ abelian} \\
        \text{If } \order{Z(G)} = p &\implies \order{G/Z(G)} = p \\
        &\implies G/Z(G) \text{ cyclic} \\
        &\implies G \text{ abelian} \implies Z(G) = G \quad \contradiction
    \end{align*}
\end{proof}

\begin{theorem}[Sylow's First Theorem]
  Let \(G\) be a finite group and let \(p\) be a prime. If \(p^k \divs \order{G}\) then \(G\) has at least one subgroup of order \(p^k\).
\end{theorem}

\begin{proof}[Proof of Sylow's First Theorem]
    Use induction on \(\order{G}\). When \(\order{G}=1\) it is trivial.

    Assume the statement holds for all groups or order less than \(\order{G}\).

    If \(H \psgp G\) and \(p^k\divs \order{H}\) then we are done by induction.

    Assume \(p^k\) does not divide the order of any proper subgroup of \(G\).

    Consider \(\order{G}=\order{Z(G)} + \sum\index{G}{C(a)}\), where we sum over a representative of each conjugacy class \(\cl{a}\) with \(a\not\in Z(G)\)

    By FTFAG (or Cauchy's theorem for abelian groups), \(\exists x \in Z(G)\) with \(\order{x}=p\)

    Since \(x\in Z(G)\implies \cyc{x}\nsgp Z(G)\nsgp G \implies \cyc{x}\nsgp G\)

    So, we can formulate \(G / \cyc{x}\)

    Since \(\order{G / \cyc{x}} = \frac{\order{G}}{\order{\cyc{x}}}=\frac{\order{G}}{p} \implies p^{k-1}\divs \order{G / \cyc{G}}\)

    Note that (\(G\to G / \cyc{x}\) is \(p\)-to-1)

    Then by induction \(\exists\)subgroup of order \(p^{k-1}\) of \(G / \cyc{x}\) and such a subgroup has form \(H / \cyc{x}\) where \(H\sgp G\).

    But now \(\order{H} / \cyc{x} = p^{k-1} and \order{\cyc{x}}=p\) so \(\order{H} = p^k\) \contradiction.
\end{proof}