\subsection*{Lecture 30 (11/04)} % mon nov 4

Recall
\begin{enumerate}
    \item If \(G\) is a finite group of permutations on a set \(S\) and \(i\in S\), then \begin{align*}
        \text{orb}_G(i) &= \{\phi(i)\st \phi\in G\} \subseteq S\\
        \text{stab}_G(i) &= \{\phi\in G\st \phi(i) = i\} \sgp G \\
        \index{G}{\stab{G}{i}} &= \order{\orb{G}{i}}
    \end{align*}
    \item (N/C Theorem) Let \(H\sgp G\). Recall the \emph{normalizer of H in G} and the \emph{centralizer of H in G}, \begin{align*}
        N_G(H)&=\{x\in G \st xHx\inv=H\} \\
        C_G(H)&=\{x\in G \st xhx\inv \in H,\ \forall h\in H\} \\
        \qg{N_G(H)}{C_G(H)}&\sgp \Aut(H)
    \end{align*}
\end{enumerate}

\begin{proof}[Proof of Sylow's Second Theorem]
    Let \(H\sgp G\), \(\order{H}=p^k\), \(p^k\divs\order{G}\)

    Let \(K\) be a Sylow \(p\)-subgroup\ of \(G\).

    Let \(C=\{K_1=K, K_2,\ldots,K_n\}\) be the set of conjugates of \(K\) by elements of \(G\)\\
    (i.e. \(K_i=\conj{g_i}{K}\) for some \(g_i\in G\))

    Then \(\order{C}=\index{G}{N_G(K)}\)

    Then the mapping \(G\to C\) where \(g\mapsto \conj{g}{K}\) is surjective.
    \begin{align*}
        \text{\(g\) and \(h\) have the same image} &\iff \conj{g}{K}=\conj{h}{K} \\
        &\iff (h\inv g)K(h\inv g)\inv = K \\
        &\iff h\inv g\in N_G(K) \\
        &\iff gN_g(K)=hN_G(K) \\
        &\iff \text{1-1 correspondence between elements of \(C\) and left coests of \(N_G(K)\)} \\
        &\implies \order{C}=\index{G}{N_G(K)}
    \end{align*}
    Consider the action of \(H\) on \(C\) given by \(h\) acts on \(K_i\) by \(\conj{h}{K_i}\)

    Then \(\order{\orb{H}{K_i}} = \index{H}{\stab{H}{K_i}}\) is a power of p and
    \begin{align*}
        \order{\orb{H}{K_i}}=1 &\iff \stab{H}{K_i}=H\\
        &\iff H\sgp N_G(K_i)
    \end{align*}
    \begin{claim}
        \(H\sgp N_G(K_i)\iff H\sgp K_i\)
    \end{claim}
    \begin{subproof}[Proof of claim]
        ``\(\impliedby\)''  obvious.

        ``\(\implies\)'' \(\forall x\in H\), \(\order{x}\) is a power of \(p\) (since \(\order{x}\divs\order{H}=p^k\))

        \(\forall y\in N_G(K_i)\sgp K_i\) \(\quad \order{yK_i}\divs\order{\qg{N_G(K_i)}{K_i}}\)

        But \(\order{\qg{N_G}{K_i}} = \frac{\order{N_G(K_i)}}{\order{K_i}}\divs\frac{\order{G}}{\order{K_i}} (\leftarrow\) this is rel prime to p since \(K_i=\) sylow p-sgp

        \(\implies p\ndivs \order{yK_i}\) and \(\order{yK_i}\neq 1\)

        \(\implies \order{y}\) is not a power of \(p\) because \(\order{yK_i}\divs \order{y}\)
    \end{subproof}
    Summing up, we see that if \(\order{\orb{H}{K_i}}=1\) then \(H\sgp K_i\).

    Now, \(\order{C}=\index{G}{N_G(K)} = \frac{\order{G}}{\order{N_G(K_i)}} = \underbrace{\frac{\frac{\order{G}}{\order{K}}}{\frac{\order{N_G(K_i)}}{\order{K}}}}_{\text{this is not divisible by } p}\).

    If no orbit of \(C\) under \(H\) has size 1, then \(p\) divides the size of each orbit

    then \(p\) divides \(\order{C}\) \contradiction\\
    (\(\implies \exists K_i \) s.t. \(\order{\orb{H}{K_i}} = 1\))
\end{proof}

\begin{proof}[Proof of Sylow's Third Theorem]
    Let \(\order{G}=p^km\) and \(K \sgp G\) be a Sylow \(p\)-subgroup \\
    Let \(C = \{K_1=K, K_2, \ldots, K_n\}\) be the set of conjugates of \(K\) in \(G\).

    Consider the action of \(K\) on \(G\) by conjugation.

    Then \begin{itemize}
        \item \(\order{\orb{K}{K_i}}=\index{K}{\stab{K}{K_i}}\) divides \(\order{K}=p^k\)
        \item \ \vspace{-2em} \begin{align*}
            \order{\orb{K}{K_i}} = 1 &\iff \stab{K}{K_1} = K \\
            &\iff K\sgp N_G(K_i) \overset{claim}{\iff} K\sgp K_i \iff K = K_i
        \end{align*}
        \(\implies n = \order{C}\) is equal to 1 modulo \(p\)

        RTS that any Sylow \(p\)-subgroup is one of the \(K_i\) (i.e. conjugate to \(K\))

        If \(K'\) is another Sylow \(p\)-subgroup of \(G\) and \(K'\not\in C\), then consider the action of \(K'\) on \(C\) by conjugation.

        Then the size of each orbit is greater than 1 (since \(\orb{K'}{K_i}=1 \iff K'=K_i\) which is impossible)

        \(\implies\) summing up, \(\order{C}\equiv 0\Mod p\) contradicting \(\order{C}\equiv 1\Mod p\)

        \(\implies\) any Sylow \(p\)-subgroup is a conjugate of \(K\) we started with.

        Finally, \(\order{C} = \frac{\order{G}}{\order{N_G(K)}}\) divides \(\order{G}=p^rm\) and \(\order{C}\equiv 1 \Mod p\).

        Since \(\gcd(p,m)=1\implies \order{C}\divs m\)
    \end{itemize}
\end{proof}
