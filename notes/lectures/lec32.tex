\subsection*{Lecture 32 (11/8)} % fri nov 8
\section{Introduction to Rings}
\subsection{Motivation \& Definition}
\begin{definition}[Ring]
  A \ul{ring} \(R\) is a set with two binary operations: \(a+b\) and \(a\cdot b=ab\) such that for all \(a,b,c\in R\),
  \begin{enumerate}
    \item \(a+b = b+a\)
    \item \((a+b)+c = a+(b+c)\)
    \item \(\exists\) an additive identity \(0\), \(a+0 = a\)
    \item \(\exists\) an element \(-a\in R\) such that \(a+(-a)=0\)
    \item \((ab)c = a(bc)\)
    \item \(a(b+c) = ab+ac\)

    \((b+c)a = ba+ca\)
  \end{enumerate}
\end{definition}

So a ring is an abelian group under addition, and also has an associative multiplication that is left and right distributive over addition.
\begin{itemize}
  \item The multiplication need not be commutative. When it is, we say the ring is commutative.
  \item A \ul{unity (or identity)}: a nonzero element that is an identity under multiplication.
  \item \ul{unit}: a nonzero element of a commutative ring with identity that has a multiplicative inverse.
  \item In \(R\), \(a\divs b\) if \(\exists c\in R\) such that \(b=ac\).
  \item \(n\in \ZZ{>0}\), \(na = \underbrace{a+a+\cdots+a}_{\text{n times}}\)
\end{itemize}

\subsection{Examples of Rings}
\begin{example}
  \((\Z, +\times)\) is a commutative ring with identity and units \(=\pm 1\)
\end{example}

\begin{example}
  \((\Z_n, +\times)\) is a commutative ring with identity and units \(=U(n)\)
\end{example}

\begin{example}
  \((\Z[x], +\times)\) is a commutative ring with identity
\end{example}

\begin{example}
  \((\M_2[\Z], +\times)\) is a non-commutative ring with identity
\end{example}

\begin{example}
  \((2\Z = \{\text{even integers}\}, +\times)\) is a comm ring without identity
\end{example}

\begin{example}
  \((\{\text{continuous functions on \R}, +\times\})\) is a comm ring with identity \(f(x) = 1\)
\end{example}

\begin{example}
  \((\{\text{continuous functions on \R\ whose graphs pass through (1, 0)}, +\times\})\) is a comm ring without identity

  Note \(f(1) = 0\), \(g(1) = 0\), \(f+g, fg\)
\end{example}

\begin{example}[Direct sum]
  Let \(R_1, R_2, \ldots, R_n\) be rings. Construct
  \begin{align*}
    R_1\edp R_2\edp \cdots \edp R_n = \{(r_1, r_2, \ldots, r_n) \mid r_i\in R_i\}
  \end{align*}
  with component-wise addition and multiplication. This ring is called the \ul{direct sum} of \(R_1, R_2, \ldots, R_n\).
\end{example}

\subsection{Properties of Rings}

\begin{theorem}[Rules of Multiplication]
  For all \(a,b,c\in R\),
\begin{enumerate}
  \item \(a\cdot 0 = 0\cdot a = 0\)
  \item \(a(-b) = (-a)b = -(ab)\)
  \item \((-a)(-b)=ab\)
  \item \(a(b-c) = ab-ac\)

  \((b-c)a = ba-ca\)
  \item \((-1)a = -a\)
  \item \((-1)(-1) = 1\)
\end{enumerate}\

\begin{note}
Properties 5 and 6 only hold if \(R\) has an identity 1
\end{note}
\end{theorem}

\begin{proof}[Proof of property 1]
    Clearly \(0+a0 = a0 = a(0+0) = a0 + a0\), so by cancellation \(0=a0\) and similarly \(0a=0\)
\end{proof}

\begin{theorem}[Uniqueness of the Unity and Inverses]
  If a ring \(R\) has a unity, it is unique. If a ring element has a multiplicative inverse, it is unique.
\end{theorem}

\begin{proof}
1, 1' \(\implies\) 1=1\(\cdot\)1' = 1'

\(a\) \(\quad\) \(ab=ba = 1\)

\(\qquad\) \(ac=ca=1\)

\(c = c\cdot 1 = c(ab) = (ca)b = 1\cdot b = b\)
\end{proof}

\begin{warning}
In general, \(ab = ac \centernot\implies b=c\) (cancellation rule does not hold in general for multiplication).
\end{warning}

\begin{example}
  In \(\Z_6\), notice \(2\cdot 3 = 0 = 3\cdot 0\) but \(2\neq 0\)
\end{example}

\subsection{Subrings}

\begin{definition}[Subring]
  A subset \(S\subseteq R\) is a \ul{subring} of \(R\) if \(S\) is itself a ring with the operations of \(R\)
\end{definition}

\begin{theorem}[Subring Test]
  A nonempty subset \(S\) of a ring \(R\) is a subring if \(S\)  is closed under subtraction and multiplication.

  i.e. if \(a,b\in S\) then \(a-b\in S\) and \(ab \in S\)
\end{theorem}

\begin{example}[Trivial Subrings]
  \(\{0\}\) and \(R\) will always be subrings of any ring \(R\).
\end{example}

\begin{example}
  \(\{0,2,4\}\subseteq \Zsi\) is a subring

  1 is the identity in \(\Zsi\)

  4 is the identity in \(\{0,2,4\}\) (\(0\cdot 4 = 0,\ 2\cdot 4 = 2,\ 4\cdot 4 = 4\))
\end{example}

\begin{example}
  \(n\Z = \{0, \pm n, \pm 2n, \pm 3n,\ldots\}\) is a subring of \(\Z\) that does not have any identity (if \(n\neq 1\)).
\end{example}