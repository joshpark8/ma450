\documentclass[a4paper]{article}
\usepackage{asymptote}

\input{preamble.tex}
\input{letterfont.tex}
\input{macros.tex}

% \renewcommand{\arraystretch}{1.25} % space out table rows
% \setlength{\parindent}{0pt}
\setlength{\parskip}{1em}
% \linespread{1} % 1.3 for one-and-half spacing, 1.6 for double spacing
\setcounter{section}{16}
\rhead{}

\begin{document}
\subsection*{Lecture 38} % mon dec 2
\section{Factorization of polynomials}
\subsection{Reducibility Tests}
\begin{definition}[Irreducible/Reducible Polynomial]
  Let \( D \) be an integral domain. A polynomial \( f(x)\in D[x] \) that is neither 0 nor a unit in \( D[x] \) is said to be \ul{irreducible} over \( D \) if whenever \( f(x) = g(x)h(x) \), then \( g(x) \) or \( h(x) \) is a unit in \( D[x] \). A nonzero, nonunit element of \( D[x] \) that is \emph{not} irreducible is said to be \ul{reducible}.
\end{definition}

\begin{example}
  \begin{align*}
    f(x) &= 2x^2 + 4 \\ &= 2\cdot (x^2+2) \\ &= 2(x+\sqrt{-2})(x-\sqrt{-2})
  \end{align*}
  Reducible over \( \Z,\ \C \). Irreducible over \( \Q,\ \R \).
\end{example}

\begin{example}
  \( x^2-2 = (x+\rtt)(x-\rtt) \) is irreducible over \( \Q \) but reducible over \( \R \).
\end{example}

\begin{theorem}[Reducibility Test for Degrees 2 and 3]
  Let \( F \) be a field and \( f(x) \in F[x] \) such that \( \deg f =\) 2 or 3. Then \( f(x) \) is reducible over \( F \)\iff \( f(x) \) has a zero in \( F \).
\end{theorem}

\begin{proof}[\bd{Pf sketch}]
  If \( f(x) = g(x)h(x) \) then \( g(x) \) or \( h(x) \) has a degree of 1 (if \( \deg g(x) = 0 \) or \( \deg h(x) = 0 \) then \( g(x) \) or \( h(x) \) is a unit).
  \begin{align*}
    \deg 1 &\imp ax+b,\quad a,b\in F \\ &\imp a(x+\frac{b}{a}) \imp \frac{-b}{a} \text{ is a zero}
  \end{align*}
\end{proof}

\begin{example}
  \( x\sq + 1 \) is irreducible over \Zth\ (\( 0\sq+1=1,\quad 1\sq+1 = 2,\quad 2\sq + 1 = 5 = 2 \) in \Zth).

  \( x\sq + 1 \) is reducible over \Zfi\ (\( x\sq+1 = (x-2)(x-3) \) in \( \Zfi[x] \)).
\end{example}

\begin{example}
  \( x^4+2x\sq + 1 = (x\sq + 1)\sq \) is reducible over \Q\ (or \R) in \( \Q[x] \) (or \( \R[x] \)) but \( x^4+2x\sq + 1 \) has no zeros in \Q\ (or in \R)
\end{example}

\begin{definition}[Content of a Polynomial, Primitive Polynomial]
The \ul{content} of a nonzero polynomial \begin{align*}
  a_nx^n + a_{n-1}x^{n-1} + \cdots + a_1x+a_0 \in \Z[x]
\end{align*}
is the greatest common divisor of \( a_n, a_{n-1}, \cdots, a_0 \). A \ul{primitive polynomials} is an element in \( \Z[x] \) with content \( 1 \).
\end{definition}

\begin{lemma}[Gauss's Lemma]
  The product of two primitive polynomials is primitive.
\end{lemma}
\end{document}