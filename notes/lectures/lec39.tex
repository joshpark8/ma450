\documentclass[a4paper]{article}
\usepackage{asymptote}

\input{preamble.tex}
\input{letterfont.tex}
\input{macros.tex}

% \renewcommand{\arraystretch}{1.25} % space out table rows
% \setlength{\parindent}{0pt}
\setlength{\parskip}{1em}
\linespread{1.3} % 1.3 for one-and-half spacing, 1.6 for double spacing
\setcounter{section}{16}
\rhead{}

\begin{document}
\subsection*{Lecture 38} % mon dec 2
\section{Factorization of polynomials}
\subsection{Reducibility Tests}
\begin{definition}[Irreducible/Reducible Polynomial]
  Let \( D \) be an integral domain.
  A polynomial \( f(x)\in D[x] \) that is neither 0 nor a unit in \( D[x] \) is said to be \tun{irreducible} over \( D \) if whenever \( f(x) = g(x)h(x) \), then \( g(x) \) or \( h(x) \) is a unit in \( D[x] \). A nonzero, nonunit element of \( D[x] \) that is \emph{not} irreducible is said to be \tun{reducible}.
\end{definition}

\begin{example}
  \begin{align*}
    f(x) &= 2x^2 + 4 \\ &= 2\cdot (x^2+2) \\ &= 2(x+\sqrt{-2})(x-\sqrt{-2})
  \end{align*}
  Reducible over \( \Z,\ \C \). Irreducible over \( \Q,\ \R \).
\end{example}

\begin{example}
  \( x^2-2 = (x+\rtt)(x-\rtt) \) is irreducible over \( \Q \) but reducible over \( \R \).
\end{example}

\begin{theorem}[Reducibility Test for Degrees 2 and 3]
  Let \( F \) be a field and \( f(x) \in F[x] \) such that \( \deg f =\) 2 or 3.
  Then \( f(x) \) is reducible over \( F \)\iff \( f(x) \) has a zero in \( F \).
\end{theorem}

\begin{proof}[\tbo{Pf sketch}]
  If \( f(x) = g(x)h(x) \) then \( \deg g(x) + \deg h(x) = \deg f(x) = \) 2 or 3.
  So \( g(x) \) or \( h(x) \) has a degree of 1 (if \( \deg g(x) = 0 \) or \( \deg h(x) = 0 \) then \( g(x) \) or \( h(x) \) is a unit).
  \begin{align*}
    \deg 1 &\imp ax+b,\quad a,b\in F \\
           &\imp a(x+\frac{b}{a}) \\
           &\imp -\frac{b}{a} \text{ is a zero of } f(x)
  \end{align*}
\end{proof}

\begin{example} \label{ex:17}
  \( x\sq + 1 \) is irreducible over \Zth \qpmi (\( 0\sq+1=1,\ 1\sq+1 = 2,\ 2\sq + 1 = 5 = 2 \) in \Zth)

  \( x\sq + 1 \) is reducible over \Zfi \qpmi (\( x\sq+1 = (x-2)(x-3) \) in \( \Zfi[x] \))
\end{example}

\begin{exercise}
  Prove \hyperref[ex:17]{Example \ref*{ex:17}}
\end{exercise}
\begin{example}
  \( x^4+2x\sq + 1 = (x\sq + 1)\sq \) is reducible over \Q\ (or \R) in \( \Q[x] \) (or \( \R[x] \)) but \( x^4+2x\sq + 1 \) has no zeros in \Q\ (or in \R)
\end{example}

\begin{definition}[Content of a Polynomial, Primitive Polynomial]
The \tun{content} of a nonzero polynomial \begin{align*}
  a_nx^n + a_{n-1}x^{n-1} + \cdots + a_1x+a_0 \in \Z[x]
\end{align*} is the greatest common divisor of \( a_n, a_{n-1}, \ldots, a_0 \).
A \tun{primitive polynomials} is an element in \( \Z[x] \) with content \( 1 \).
\end{definition}

\begin{lemma}[Gauss's Lemma]
  The product of two primitive polynomials in \( \Z[x] \) is primitive.
\end{lemma}

\begin{proof}
  Assume \( f(x),g(x) \) are primitive, and suppose \( f(x)g(x) \) is not primitive.
  Let \( p \) be a prime divisor of the content of \( f(x)g(x) \).
  Consider the ring homomorphism from \( \phi:\Z[x]\to \ZZ p[x] \).
  Let \( \bar{f(x)} \bar{g(x)} \) be the image of \( f(x)g(x)  \) in \( \ZZ p[x] \) \imp \( \bar{f(x)g(x)} = \bar{f(x)}\bar{g(x)} \)
  \begin{note}
    In other words, \( \bar{f(x)} \) is the polynomial in \( \Z[x] \) obtained by reducing the coefficients of \( f(x) \) modulo \( p \).
  \end{note}
  Since \( p\divs\) content of \( f(x)g(x) \) \imp \( \bar{f(x)}\bar{g(x)}=0 \) in \( \ZZ p[x] \)\\
  \imp \( \bar{f(x)} = 0 \) or \( \bar{g(x)} = 0 \) because \( \ZZ p[x] \) is an integral domain. \\
  \imp \( f(x) \) or \( g(x) \) is not primitive. \contradiction
\end{proof}

\begin{theorem}
  Let \( f(x)\in\Z[x] \). If \( f(x) \) is reducible over \Q, it is reducible over \Z.
\end{theorem}

\begin{proof}
  Assume \( f(x)=g(x)h(x) \) with \( g(x),h(x)\in\Q[x] \).
  Let \( a \) and \( b \) be the LCM of denominators of coefficients of \( g(x) \) and \( h(x) \) respectively.
  Then \( (ab)f(x) = abg(x)h(x) = (ag(x))(bh(x)) \).
  Let \( c_1 \) and \( c_2 \) be the content of \( ag(x) \) and \( bh(x) \) respectively.
  Then \( ag(x) = c_1\hat g(x) \) and \( bh(x) = c_2\hat h(x) \) where \( \hat g(x) \) and \( \hat h(x) \) are primitive in \( \Z[x] \).
  Let \( d \) be the content of \( f \) (i.e. \( f(x) = d\hat f(x) \) where \( \hat f(x) \in \Z[x]\) is primitive.)
  Then \( (abd)\hat f(x) = (c_1c_2)\hat g(x)\hat h(x) \in \Z[x] \).
  By Gauss' lemma, \( \hat g(x) \hat h(x) \) is primitive in \( \Z[x] \)\\
  \imp \( abd = c_1c_2 \) \imp \( \hat f(x) = \hat g(x)\hat h(x) \) \\
  \imp \( f(x) = d\hat f(x) = (d\hat g(x))\cdot\hat h(x) \) \\
  \imp \( f(x) \) is reducible over \( \Z \) (since \( d\hat g(x),\hat h(x)\in \Z[x] \)).
\end{proof}

\begin{example}
  \( f(x) = 6x\sq + x - 2 = (\underbrace{3x-\frac{3}{2}}_{g(x)})(\underbrace{2x+\frac{4}{3}}_{h(x)}) \)

  \( d=1,\ a=2,\ b=3,\ c_1=3,\ c_2=2 \) \imp \( f(x) = (2x-1)(3x+2) \)

  FINISH EXAMPLE (NOTES-38)
\end{example}

\begin{theorem}
  Let \( p \) be prime and \( f(x)\in\Z[x] \) such that \( \deg f \geq 1 \).
  \( \bar{f(x)} \) reducing coeff of \( f(x) \) modulo \( p \).

  If \( \bar{f(x)} \) is irreducible over \( \ZZ p \) and \( \deg \bar{f(x)} = \deg f(x) \), then \( f(x) \) is irreducible over \( \Q \).
\end{theorem}

\begin{remark}
  \( f(x) = 21x^3-3x^2+2x+9 \) work over \( \Ztw \)

  \( \bar{f(x)} = x\cb+x\sq+1 \) has no zero in \( \Ztw \) \imp irriducible
\end{remark}

\end{document}