% !TeX root=../group theory notes.tex

\subsection*{Lecture 24 (10/21)} % mon oct 21
\section{Group Homomorphisms} % ch 10

\begin{definition}[homomorphism]
  A \emph{homomorphism} \(\phi: G\to \bar G\) between two groups is a mapping that preserves the group operation:
  \begin{align*}
      \phi(ab) = \phi(a)\phi(b)\quad \forall a,b\in G.
  \end{align*}
  \end{definition}

  \begin{definition}[kernel]
  The \emph{kernel} of a homomorphism \(\phi: G\to \bar G\) is the set
      \begin{align*}
          \ker(\phi) = \{x\in G \mid\phi(x)=\bar e\}.
      \end{align*}
  \end{definition}

  \begin{example}
      Any isomorphism is a homomorphism with \(\ker\phi=\{e\}\).
  \end{example}

  \begin{examples}
  \begin{itemize}
      \item \(\phi:\GL(2,\bbR)\to (\bbR^*, \cdot)\) where \(A\mapsto \det(A)\).

      Then \(\phi(AB) = \det(AB) = \det(A)\det(B) = \phi(A)\phi(B)\) and \(\ker\phi = \SL(2,\bbR)\).

      \item \(\phi: \Z \to \Z_n\) where \(x\mapsto x\Mod n\).

      Then \(\ker\phi = \cyc{n} = n\Z\)

      \item \(\phi: (\bbR^*, \cdot)\to (\bbR^*, \cdot)\) where \(x\mapsto x^2\).

      Then \(\phi(xy)=(xy)^2 = x^2y^2=\phi(x)\phi(y)\) and \(\ker\phi = \{-1, 1\}\)
  \end{itemize}
  \end{examples}

  \begin{nonexamples}
  \begin{itemize}
      \item \(\phi:(\bbR,+)\to (\bbR,+)\) where \(x\mapsto x^2\). Notice that
      \begin{align*}
          \phi(x+y)&=(x+y)^2 \\
          \neq \phi(x)+\phi(y) &= x^2+y^2
      \end{align*} so \(\phi\) is \uline{not} a homomorphism.

      \item \(\phi:\ZZ{3}\to\ZZ{6}\) where \(x\mapsto 3x\Mod 6\)
      \begin{align*}
          \phi(x+y) &= [3(x+y\Mod 3)]\Mod 6 \\
          \phi(x)+\phi(y) &= [(3x\Mod 6) + (3y\Mod 6)]\Mod 6
      \end{align*}
      Now let \(x=1\) and \(y=2\). Then \(\phi(1+2)=0\) but \(\phi(x)+\phi(y)=(3+0)\Mod 6 = 3\). Thus \(\phi\) is \uline{not} a homomorphism
  \end{itemize}
  \end{nonexamples}

  \begin{theorem}[Properties of elements under homomorphism]
      Let \(\phi: G\to\bar G\) be a homomorphism. Then
      \begin{enumerate}
          \item \(\phi(e)=\bar e\)
          \item \(\phi(g^n)=\phi(g)^n\quad \forall g\in G\)
          \item \(\abs{g}\text{ finite} \implies \abs{\phi(g)} \divs \abs{g}\)
          \item \(\ker\phi \sgp G\)
          \item \(\phi(a) = \phi(b) \iff a\cdot\ker\phi = b\cdot\ker\phi\)
          \item \(\phi(g)=g'\implies \phi\inv(g')=\{x\in G \mid\phi(x)=g'\} = g\cdot \ker\phi\)
      \end{enumerate}
  \end{theorem}

  \begin{example}
      Any \homo\ \(\phi_i: \ZZ{3}\to\ZZ{6}\) is determined by \(\phi(1)\).

      Note that \(\abs{\phi (1)} \divs \abs{1} = 3 \implies \abs{\phi (1)}=1 \text{ or } \abs{\phi (1)}=3\)
      \begin{align*}
          \abs{\phi(1)} = 1 \quad &\implies \quad \phi(1)=0 \quad \implies \quad \phi(x)=0\ \forall x \quad \text{(i.e. \(\phi\) is the trivial \homo)} \\
          \abs{\phi(1)} = 3 \quad &\implies \quad \phi(1)=2 \text{ or } \phi(1)=4 \\
          \phi(1) = 2 \quad &\implies \quad \phi(x)=2x \Mod 6 \\
          \phi(1) = 4 \quad &\implies \quad \phi(x)=4x \Mod 6
      \end{align*}
  \end{example}

  \begin{example}
      Any \homo\ \(\phi_i: \Z_m\to\Z_n\) is determined by \(\phi(1)\).
       \begin{align*}
          \left.
             \begin{array}{l}
              \abs{\phi(1)} \divs m \\ \abs{\phi(1)} \divs n
          \end{array}
          \right\}\implies \abs{\phi(1)} \divs \gcd(m,n)
       \end{align*}
  \end{example}

  \begin{exercise}
    For all \(g\in\Z_n\) with \(\abs{y}\divs\gcd(m,n)\), $\exists$hom. \(\phi:\Z_m\to\Z_n\) sending 1 to \(y\) (so, \(\phi(x)=xy\Mod n\)).
  \end{exercise}

  \begin{theorem}[Properties of sgps under \homo s]
      Let \(\phi: G\to\bar G\) be a homomorphism and \(H\sgp G\). Then
      \begin{enumerate}
          \item \(\phi(H)=\{\phi(h)\mid h\in H\}\) is a sgp of \(\bar G\)
          \item \(H\) cyclic \(\implies \phi(H)\) cyclic
          \item \(H\) abelian \(\implies \phi(H)\) abelian
          \item \(H\) normal \(\implies \phi(H) \nsgp \phi(G)\)
          \item \(\abs{\ker\phi} = n \implies \phi\) is an n-to-1 mapping from \(G\) onto \(\phi(G)\)
          \item \(\abs{H} = n \implies \abs{\phi(H)} \divs n\)
          \item \(\Kbar \sgp \Gbar \implies \phi\inv(\Kbar)=\{k\in G\mid\phi(k)\in\Kbar\}\sgp G\)
          \item \(\Kbar \nsgp \Gbar \implies \phi\inv(\Kbar) \nsgp G\) \\
          (\(\implies\) \textbf{Cor:} \(\ker\phi = \phi\inv(\bar{e}) \nsgp G\))
          \item \(\phi\) is injective \(\iff\) \(\ker\phi=\{e\}\) \\
          \(\phi\) is an isomorphism \(\iff\) \(\phi\) is onto and \(\ker\phi=\{e\}\)
      \end{enumerate}
  \end{theorem}

  \begin{examples}
    \begin{itemize}
      \item \(\phi:\ZZ{3}\to\ZZ{6}, \quad\phi(1) = 4 \implies \phi(2)=2,\ \phi(0)=0\) \\
      \(\implies\ker\phi = \{0\}\). \(\phi\) is 1-1 but not onto.
      \item \(\phi:\Z_{12}\to\Z_{12}, \quad \phi(1)=3 \implies \phi(x)=3x \Mod 12\) \\
      \(\implies \ker\phi = \{0, 4, 8\}\implies \phi\) is 3-to-1 mapping
      e.g. \begin{align*}
        \phi(2) = 6\implies \phi\inv(6) &= 2+\{0, 4, 8\} \\
        &= \{2,6,10\} \\
        \phi\inv(\cyc{6}) = \phi\inv(\{0,6\}) &= \{0,2,4,6,8,10\} \\
        &= \cyc{2} \sgp \Z_{12}
      \end{align*}
    \end{itemize}
  \end{examples}

  \begin{theorem}[First Isomorphism Theorem]
      Let \(\phi:G\to\Gbar\) be a group \homo. Then, the mapping \(G/\ker\phi\mapsto \phi(G)\) where \(g\cdot\ker\phi\mapsto \phi(g)\) is an isomorphism. That is, \(G/\ker\phi\cong \phi(G)\).
  \end{theorem}