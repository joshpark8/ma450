\subsection*{Lecture 36} % wed nov 20

\section{Ring Homomorphisms}
\subsection{Definition and Examples}
  \begin{definition}[Ring Homomorphism, Ring Isomorphism]
    A \ul{ring homomorphism}\  \( \phi: R\to S \) is a map that preserves the two operations:
    \begin{enumerate}
      \item \( \phi(a+b) = \phi(a) + \phi(b) \)
      \item \( \phi(ab) = \phi(a)\phi(b) \)
    \end{enumerate}
    A bijective ring homomorphism is called a \ul{ring isomorphism}.
  \end{definition}

\begin{examples}\
  \begin{itemize}
    \item \( \phi:\Z\to\Zn \), \( k\mapsto k\Mod n \)
    \item \( \phi:\C\to\C \), \( a+bi\mapsto a-bi \) (isomorphism)
    \item \( \phi:\R[x]\to\R \), \( f(x)\mapsto f(a) \) where \( a\in\R \)
    Check that \( \phi(f(x) + g(x)) = \phi(f(x)) + \phi(g(x)) \) and \( \phi(f(x)g(x)) = \phi(f(x))\phi(g(x)) \)
  \end{itemize}
\end{examples}

\begin{example}
  \( \phi:\Zfo\to\Zte,\ x\mapsto 5x \)
  \begin{align*}
    (!!!)\quad \phi(x+y) &= 5(x+y\Mod 4) \Mod{10} \\
    &= 5x+5y = \phi(x)+\phi(y) \\
    (\bigstar)\quad \phi(xy) &= 5xy \Mod{10} \\
    &= 5x 5y \Mod{10} = \phi(x)\phi(y)
  \end{align*}
\end{example}

\begin{example}
  Determine all ring homomorphisms \( \ZZ{12}\mapsto\ZZ{30} \)

  Group homomorphisms: \( x\mapsto ax \) where \( \order{a}\divs\gcd(12,30) = 6 \) (i.e., \( \order{a} = 1,\ 2,\ 3,\text{ or } 6 \))

  \( \imp \) \( a = 0, 15, 10, 20, 5, 25 \)

  Ring homomorphisms: \( a = \phi(1) = \phi(1\cdot 1) = \phi(1)\phi(1) = a^2\Mod 30\)

  \( \imp \) \( a\equiv a^2 \Mod 30 \)

  \( \imp \) \( a\neq 5,\ a\neq 20 \) (\( \phi(xy) = axy = a^2xy = axay = \phi(x)\phi(y) \Mod 30\))

  Thus there are 4 ring homomorphisms:
  \begin{align*}
    x\mapsto 0x \Mod 30 \qquad x\mapsto 15x \Mod 30 \qquad x\mapsto 10x \Mod 30 \qquad x\mapsto 25x \Mod 30
  \end{align*}
\end{example}

\begin{example}
  \( R \) commutative ring, \( \char{R} = p > 0 \)

  \( \phi: R \to R \), \( x\mapsto x^p \)
  \begin{align*}
    \phi(xy) &= (xy)^p = x^py^p = \phi(x)\phi(y) \\
    \phi(x+y) &= (x+y)^p = x^p + y^p + \underbrace{\sum_{i=1}^{p-1} \binom{p}{i} x^i y^{p-i}}_{p \text{ divides } \binom{p}{i}} = x^p + y^p = \phi(x) + \phi(y)
  \end{align*}
\end{example}

\subsection{Properties of Ring Homomorphisms}
\begin{theorem}[Properties of Ring Homomorphisms]
  Let \( \phi: R\to S \) be a ring homomorphism. Then
  \begin{enumerate}
    \item \( \phi(nr) = n\phi(r),\  \phi(r^n) = \phi(r)^n \quad \forall r\in R, n\in \ZZ{>0} \)
    \item \( A \) is a subring of \( R \implies  \phi(A) = \{\phi(a) \st a\in A\} \) is a subring of \( S \)
    \item \( A \) ideal and \( \phi \) onto \( S \) \( \imp \phi(A) \) ideal of \( S \)
    \item \( \phi\inv(B) = \{r\inR \st \phi(r) \in B \} \) is an ideal of \( R \)
    \item If \( R \) commutative, then \( \phi(R) \) commutative
    \item \hspace{-2.75em}$\bigstar$\hspace{1.5em} If \( R \) has a unity 1, \( S\neq \{0\} \), and \( \phi \) is onto, then \( \phi(1) \) is the unity of \( S \).
    \item \( \phi \) is an isomorphism \iff \( \phi \) is onto and \( \kerphi = \{r\in\R \st \phi(r) = 0\} = \{0\} \).
    \item If \( \phi \) is an isomorphism from \( R \) onto \( S \), then \( \phi\inv \) is an isomorphism from \( S \) onto \( R \).
  \end{enumerate}
\end{theorem}