\subsection*{Lecture 35} % mon nov 18
\subsection{Prime Ideals and Maximal Ideals}

\begin{definition}[Prime Ideal, Maximal Ideal]
  A \ul{prime ideal} \( P \) of a commutative ring \( R \) is a proper ideal of \( R \) such that if \( a,b\in R \) and \( ab\in P \), then \( a\in P \) or \( b\in P \)

  A \ul{maximal ideal} of a commutative ring \( R \) is a proper ideal \( A \) of \( R \) such that if \( B \) is an ideal of \( R \) and \( A\subseteq B\subseteq R \), then \( B=A \) or \( B=R \)
\end{definition}

\begin{example}
  \( n\Z \sseq \Z \) is a prime ideal \( \iff n=0 \) or \( n \) prime
  \begin{note}
    \( n = 0 \), if \( a,b\in \Z \) such that \( ab = 0, \) then \( a=0 \) or \( b=0 \) \checkmark

    \( n \) prime, if \( a,b\in\Z \), \( n\divs ab \) then \( n\divs a  \) or \( n\divs b \) \checkmark
  \end{note}
  Moreover, \( n\Z\sseq\Z \) is a maximal ideal \tiff \( n \) prime.
\end{example}

\pagebreak