\subsection*{Lecture 33 (11/13)} % wed nov 13

\begin{example}
  The set of Gauss integers \(\Z[i] = \{a+bi\mid a,b\in\Z\}\) is a subring of \(\C\).
\end{example}

\section{Integral Domains}
\subsection{Definition and Examples}
\begin{definition}[Zero-Divisors]
  A \uline{zero-divisor} is a nonzero element \(x\) of a commutative ring \(R\) such that there is
  a nonzero element \(y\in R\) with \(xy=0\).
\end{definition}

\begin{example}
  In \(R = \Zsi\), \(x=2\) is a zero-divisor
\end{example}

\begin{definition}[Integral Domain]
  An \uline{integral domain} is a commutative ring with unity and no zero-divisors.
\end{definition}

Thus, in an integral domain, \(ab=0\implies a=0\) or \(b=0\).

\begin{example}
  The ring of integers \(\Z\) is an integral domain.
\end{example}

\begin{example}
  The ring of Gaussian integers \(\Z[i] = \{a+bi \mid a,b\in\Z\}\) is an integral domain.
\end{example}

\begin{example}
  The ring \(\Z[x]\) of polynomials with integer coefficients is an integral domain.
\end{example}

\begin{example}
  The ring \(\Z[\sqrt{2}]=\{a+b\sqrt{2}\mid a,b\in\Z\}\) is an integral domain.
\end{example}

\begin{example}
  The ring \(\ZZ p\) where \(p\) is prime is not an integral domain.
\end{example}

\begin{nonexample}
  The ring \(\ZZ n\) where \(n\) is not prime is not an integral domain.
  \begin{note}
    Write \(n=ab\) where \(1<a,b<n\) \(\implies\) \(a,b\) are both zero-divisors in \(\ZZ n\).
  \end{note}
\end{nonexample}

\begin{nonexample}
  The ring \(\Z \edp \Z\) is not an integral domain.
  \begin{note}
    \((1,0)\times(0,1) = (0,0)\)
  \end{note}
\end{nonexample}

\begin{theorem}[Cancellation]
  Let \(R\) be an integral domain. If \(a\neq 0\), then \(ab=ac\implies b=c\)
\end{theorem}

\begin{proof}
  \(ab = 0,\quad a\neq 0 \implies 0=a\inv ab = b\)
\end{proof}

\subsection{Fields}
\begin{definition}[Field]
  A \uline{field} is a commutative ring with unity in which every nonzero element is a unit
\end{definition}

\begin{fact}
  Every field is an integral domain.
\end{fact}

\begin{examples}
  \(\C,\ \bbR,\ \Q,\ \ZZ p\)
  \begin{note}[\(\ZZ p\)]
    \(1\leq a < p\) then \(gcd(a,p) = 1\); \(as+pt=1 \implies as = 1 \Mod p \implies a\) is a unit in \(\ZZ p\)
  \end{note}
\end{examples}

\begin{nonexamples}
  \(\Z,\ \Z[i]\)
\end{nonexamples}

\begin{theorem}
  A finite integral domain is a field.
\end{theorem}

\begin{proof}
\(a\in R\) if \(a = 1 \implies a\inv = 1\)

Suppose \(a\neq 1\). Consider \(a, a^2, a^3, \ldots\)

\(R\) is finite \(\implies \exists i>j\) such that \(a^i = a^j\)

\(a^i = a^j\cdot a^{i-j} \implies a^{i-j} = 1 \implies a\cdot(a^{i-j-1}) = 1 \implies a\inv =a^{i-j-1}\) exists in \(R\).
\end{proof}

\begin{example}
  \(\Zth[i] = \{a+bi \mid a,b\in \Zth\}\) is a field with 9 elements.

  \((a+bi)\inv = \frac{a-bi}{a\sq+b\sq}\) need to check if \(a,b\in\Zth \) then \(a\sq+b\sq \neq 0\) in \(\Zth\) (unless \(a=b=0\)).

  \((1+2i)\inv\) in \(\Zth[i]\) is \( \frac{1-2i}{1+4} = (1-2i)\cdot 2\inv = 2(1+1\cdot i) = 2+2i\)
\end{example}

\begin{example}
  \( \Q[\sqrt 2] = \{a+b\rtt \mid a,b\in\Q \}\)
  is a field.
  \begin{align*}
    (a+b\rtt)\inv &= \frac{a-b\rtt}{(a+b\rtt)(a-b\rtt)} = \frac{a-b\rtt}{a\sq-2b\sq} \\
    &= \frac{a}{a\sq-2b\sq} - \frac{b}{a\sq-2b\sq}\rtt \quad (a\sq-2b\sq \neq 0)
  \end{align*}
\end{example}

\begin{definition}[Characteristic]
  The \uline{characteristic} of a ring \( R \) is the least positive integer \( \char R = n \) such that \( \underbrace{nx}_{\sum^n x} = 0 \) for all \( x\in R \). If no such integer exists, we say \( R \) has characteristic 0.
\end{definition}

\begin{examples}
  \( \char{\Z}=0,\ \char{\ZZ n} = n,\ \char{\Ztw} = 2 \)
\end{examples}

\begin{theorem}
  Let \( R \) be a ring with unity 1.
  If 1 has infinite order under addition, then \( \char R = 0 \).
  If 1 has order \( n \) under addition, then \( \char R = n \)
\end{theorem}

\begin{proof}
  \(n\cdot 1 =0 \implies n\cdot x = \sum^n x = x\cdot\sum^n 1 = x\cdot 0 = 0\)
\end{proof}

\begin{theorem}
  If \( R \) is an integral domain, then \( \char R \) is either 0 or prime.
\end{theorem}

\begin{proof}
  Suppose \( \char{R} = n\geq 0 \iff\) 1 has finite order n under addition by Thm.
  If \( n=st \) where \( 1<s,t<n \), then
  \begin{align*}
    0=n\cdot 1 = (s\cdot 1)(t\cdot 1)
  \end{align*}
  so \( s\cdot 1 = 0 \) or \( t\cdot 1 = 0 \).
  Since \( \char{1} = n \), it must be that \( s=n \) or \( t=n \). However, \( s,t<n \).
\end{proof}

\section{Ideals and Factor Rings}

\subsection{Ideals}

\begin{definition}[Ideal]
  A subring \( I \) of a ring \( R \) is called a (two-sided) \uline{ideal} of \( R \) if \( \forall r\in R, \forall a\in I \) we have \( ra \in I \) and \( ar\in I \)
\end{definition}

\begin{itemize}
  \item So a subring of \( R \) is an ideal if it ``absorbs'' elements of \( R \)
  \item An ideal of \( R \) is called a \uline{proper ideal} if \( I\neq R \)
\end{itemize}

\begin{theorem}[Ideal Test]
  A nonempty subset \( I \) of a ring \( R \) is an ideal if
  \begin{enumerate}
    \item \( a-b\in I \) whenever \( a,b\in I \)
    \item \( ra \) and \( ar \) are in \( I \) for all \( a\in I \) and for all \( r\in R \)
  \end{enumerate}
\end{theorem}

\begin{example}
  For any ring \( R \), \( \{0\} \) and \( R \) are ideals.
\end{example}

\begin{example}
  \( n\Z \) is an ideal of \( \Z \) for all \( n\in\Z \)
\end{example}

\begin{example}
  \( \cyc{a}:=\{ra\mid r\in R\} \) is an ideal of \( R \) for all commutative rings with unity and \( a\in R \). This is called the \uline{principal ideal} generated by \( a \).
\end{example}

\begin{example}
  \( R=\bbR[x]\quad I=\cyc{x}= \{\text{polynomials with constant term 0}\} \)
\end{example}

\begin{example}
  Let \( R \) be a commutative ring with unity, \( a_1,a_2,\ldots,a_n\in R \). Then
  \begin{align*}
    I = \lt\{\sum^n  r_ia_i \mid r_i\in R\rt\}
  \end{align*}
  is an ideal of \( R \), called the ideal generated by \( a_1,a_2,\ldots,a_n\in R \).
\end{example}
