\documentclass{article}

\input{preamble}
\input{letterfont}
\input{macros}


% Misc helpers
% \setlength{\parskip}{1em}
\setlength{\parindent}{0pt}
% \setlist[enumerate]{itemsep=-2mm}
\setlist[itemize]{itemsep=0mm}
\renewcommand{\arraystretch}{1.25} % space out table rows

% Title
\fancyhead[L]{Josh Park}
\fancyhead[C]{\bd{MA 45000 Homework 13}}
\fancyhead[R]{Fall 2024}

% Document
\begin{document}
\bd{Exercise 13.4.}
List all zero-divisors of \( \ZZ{20} \). Can you see a relationship between the zero-divisors of \( \ZZ{20} \) and the units of \( \ZZ{20} \)?

\begin{solution} % 13.4
  By def zero-divisor, we wish to find all \( a_{\neq 0}\in R = \ZZ{20} \) such that \( \exists b_{\neq 0} \in R \) where \( ab\equiv 0 \pmod{20} \).
  That is, we wish to find all \( a_{\neq 0} \in R \) such that \( ab = 20n \) for some \( n\in\Z \) where \( b_{\neq 0}\in R \).
  % The prime factors of \( 20 = 2^2 \cdot 5 \) so we can say
  We can rewrite this as
  \begin{align*}
    ab = 20n \qimp \frac{ab}{20} = n.
  \end{align*}

  Suppose \( a \) is coprime to \( 20 \). Then by def coprime, \( a \) and \( 20 \) share no common factors.
  So \( 2\ndivs a \) and \( 5\ndivs a \) which implies \( 2p+5q\ndivs a\ \forall p,q\in\Z \).
  That is, \( a \) is not divisible by any linear combination of 2 and 5 with integer coefficients, and consequently by any divisor (nor by any multiple) of \( 20 \). We know \( a \) is an integer, so
  \begin{align*}
    n = \frac{ab}{20} = a \cdot \frac{b}{20}\in\Z \qiff \frac{b}{20}\in\Z.
  \end{align*}
  Then \( b \equiv 0\pmod{20} \), but \( b\not\equiv 0 \) by def \( b \) \contradiction. Thus \( a \) must not be coprime to \( 20 \).

  Suppose \( a \) is \emph{not} coprime to \( 20 \). Then by def coprime, \( a \) shares at least one common factor with \( 20 \). Let this factor be \( p \). Then \( a = pq \) and \( 20 = pr \) for some \(q,r \in\ZZ{20} \). Suppose \( b = r \). Then,
  \begin{align*}
    ab = 20n \qiff pqr = prn \qiff r = n
  \end{align*}
  We know \( r\in\Z \), so all numbers not coprime to 20 in \( \ZZ{20} \) are zero-divisors.

  So we have that \( a \) coprime to 20 \imp \( a \) not zero-divisor and \( a \) not coprime to 20 \imp \( a \) is zero-divisor. That is, \( a\in\ZZ{20} \) is a zero divisor \iff\ \( a \) is not coprime to \( 20 \). Thus the set of all zero divisors of \( \ZZ{20} \) is \{2, 4, 5, 6, 8, 10, 12, 14, 15, 16, 18\}.

  The set of zero-divisors of \( \ZZ{20} \) and the set of units of \( \ZZ{20} \) are disjoint and form a partition of \( \ZZ{20} \).
\end{solution}

\bd{Exercise 13.24.}
Find a zero-divisor in \( \Zfi[i] = \{a+bi\st a,b\in\Zfi\} \).

\begin{solution} % 13.24
  Let \( R = \Zfi[i] \). By def zero-divisor, \( r_{\neq 0}\in R \) is a zero-divisor of \( R \) if there exists some \( s_{\neq 0}\in R \) such that \( rs \equiv 0 \pmod 5 \). Consider the elements \( r = 2+i \) and \( \bar{r} = 2-i \). Notice
  \begin{align*}
    rs = (2+i)(2-i) = 4 -2i + 2i + 1 = 5 + 0i \equiv 0 \pmod 5
  \end{align*}
  Thus \( r \) is a zero-divisor of \( \Zfi[i] \).
\end{solution}

\bd{Exercise 13.30.}
Let \( d \) be a positive integer. Prove that \( \Q[\sqrt d] = \{a + b\sqrt d\st a,b\in\Q\}\) is a field.

\begin{solution} % 13.30
  Viewed as an element of \( \bbR \), the multiplicative inverse of any element of the form \( a+b\sqrt d \) is \( 1/(a+b\sqrt d) \). To verify that \( \Q[\sqrt d] \) is a field, we must show \( 1/(a+b\sqrt d) \) can be written in the form \( \alpha+\beta\sqrt d \).
  \begin{align*}
    \frac{1}{a+b\sqrt d} = \frac{1}{a+b\sqrt d} \cdot \frac{a-b\sqrt d}{a-b\sqrt d} = \frac{a-b\sqrt{d}}{a\sq -ab\sqrt{d} + ab \sqrt{d} - b\sq d} = \frac{a}{a\sq-b\sq d} - \frac{b}{a\sq-b\sq d}\sqrt{d}
  \end{align*}
  Thus \( \Q[\sqrt{d}] \) is a field.
\end{solution}

\bd{Exercise 13.31.}
Let \( R \) be a ring with unity 1. If the product of any pair of nonzero elements of R is nonzero, prove that \( ab=1 \) implies \( ba=1 \).

\begin{solution} % 13.31
  We have that \( a_{\neq 0}, b_{\neq 0}\in R \imp ab \neq 0 \). Suppose \( ab = 1 \). Then \begin{align*}
    ab &= 1 \\
    aba &= a \\
    aba - a &= 0 \\
    a(ba - 1) &= 0
  \end{align*}
    Notice that \( a \) is nonzero, so \( ba - 1 = 0 \) and thus \( ba = 1 \).
\end{solution}

\bd{Exercise 13.32.}
Let \( R = \{0, 2, 4, 6, 8\} \) under addition and multiplication modulo 10. Prove that \( R \) is a field.

\begin{solution} % 13.32
  By def field, we need only verify each nonzero element of \( R \) has a multiplicative inverse. The nonzero elements of \( R \) are \{2, 4, 6, 8\}. By Exercise 12.2, we know the unity of \( R \) is 6. Thus, we must find some \( b\in R \) for each \( a\in\R \) such that \( ab = 6 \). Then, we can see that
  \begin{align*}
    &2\cdot 8 = 16 \equiv 6 \pmod{10}, &4\cdot 4 = 16 \equiv 6 \pmod{10}, \\
    &6\cdot 6 = 36 \equiv 6 \pmod{10}, &8\cdot 2 = 16 \equiv 6 \pmod{10}.
  \end{align*}
  Thus \( R \) is a field.
\end{solution}

\bd{Exercise 13.42.}
Construct a multiplication table for \( \Ztw[i] \), the ring of Gaussian integers modulo 2. Is this ring a field? Is it an integral domain?

\begin{solution} % 13.42
  We know \( \Ztw[i] = \{a+bi\st a,b\in\Ztw\} = \{0, i, 1, 1+i\} \)

  Then the multiplication table is
  \begin{center}
    \begin{tabular}{c | c c c c c}
                  & 0       & \( i \)   & 1         & \( 1+i \) \\ \cline{1-5}
        0         & 0       & 0         & 0         & 0  \\
        \( i \)   & 0       & 1         & \( i \)   & \( 1+i \)  \\
        1         & 0       & \( i \)   & 1         & \( 1+i \)  \\
        \( 1+i \) & 0       & \( 1+i \) & \( 1+i \) & 0
    \end{tabular}
  \end{center}
  Since \( (1+i)\sq = 0 \), it is a zero-divisor of \( \Ztw[i] \) by def zero-divisor. Thus \( \Ztw[i] \) is not an integral domain by def integral domain.
  Thus \( \Ztw[i] \) is not a field by def field.
\end{solution}

\bd{Exercise 13.43.}
The nonzero elements of \( \Zth[i] \) form an abelian group of order 8 under multiplication. Is it isomorphic to \( \Zei,\ \Zfo\edp\Ztw, \) or \( \Ztw\edp\Ztw\edp\Ztw \)?

\begin{solution} % 13.43
  We know \begin{align*}
    \Z_3[i] = \{a+bi \st a,b\in\Zth\} = \{0, i, 2i, 1, 1+i, 1+2i, 2, 2+i, 2+2i\},
  \end{align*}
  so let \( G = \{i, 2i, 1, 1+i, 1+2i, 2, 2+i, 2+2i\} \).
  By thm, a group isomorphism must preserve the order of elements of the group.
  Thus we can test the orders of the elements of \( G \) to find an isomorphism.
  Consider the element \( \alpha=1+i \).
  Notice, \( (1+i)^2 = 2i \equiv -i \pmod 3 \), so \( (1+i)^4 = -1 \) and \( \order{\alpha} \) has order 8.
  By thm, the order of an element of an external direct product is the LCM of the orders of the elements.
  Then \( \Zfo\edp\Ztw \) and \( \Ztw\edp\Ztw\edp\Ztw \) can not have any elements of order 8, but \( \Zei \) can.
  Thus the set of nonzero elements of \( \Zth[i] \) is isomorphic to \( \Zei \).
\end{solution}

\begin{note}[Notation]
  I will use \( \srg \) to denote subring and \( \idl \) for ideal.
\end{note}

\bd{Exercise 14.4.}
Find a subring of \( \Z\edp\Z \) that is not an ideal of \( \Z\edp\Z \).

\begin{solution} % 14.4
Consider the set \( R = \{(x,x)\st x\in\Z\} \).
Notice \begin{align}
  (\alpha,\alpha) - (\beta,\beta) &= (\alpha-\beta, \alpha-\beta) \in R \label{srt c1}\\
  (\alpha,\alpha) \cdot (\beta,\beta) &= (\alpha\beta,\alpha\beta) \in R,
\end{align}
so \( R\sgp\Z\edp\Z \) by the subring test. Consider the elements \( a=(\alpha,\alpha)\in R \) and \( r = (\beta, \gamma) \in \Z\edp\Z \) such that \( \beta\neq\gamma \). Then,
\begin{align*}
  ar = (\alpha,\alpha)\cdot (\beta,\gamma) = (\alpha\beta, \alpha\gamma).
\end{align*}
We know \( \beta\neq\gamma \), so \( \alpha\beta\neq\alpha\gamma \). Then \( (\alpha\beta,\alpha\gamma) \notin R \) whence \( R\notidl \Z\edp\Z \) by the ideal test.


\end{solution}

\bd{Exercise 14.6.}
Find all maximal ideals in

\bd{a. } \Zei \qquad \bd{b. } \ZZ{10} \qquad \bd{c. } \ZZ{12} \qquad \bd{d. }\Zn

\begin{solution} % 14.6
\end{solution}

\bd{Exercise 14.10.}
If \( A \) and \( B \) are ideals of a ring, show that the \ital{sum} of \( A \) and \( B \), \( A+B = \{a+b\st a\in A,b\in B\} \), is an ideal.

\begin{solution} % 14.10
\end{solution}


\bd{Exercise 14.11.}
In the ring of integers, find a positive integer \( a \) such that
\begin{enumerate}[label=\bd{\alph*.}]
  \item \( \cyc{a} = \cyc{2} + \cyc{3} \)
  \item \( \cyc{a} = \cyc{6} + \cyc{8} \)
  \item \( \cyc{a} = \cyc{m} + \cyc{n} \)
\end{enumerate}

\begin{solution} % 14.11
\end{solution}

\bd{Exercise 14.12.}
If \( A \) and \( B \) are ideals of a ring, show that the \ital{product} of \( A \) and \( B \), \( AB = \{a_1b_1 + a_2b_2 + \cdots + a_nb_n \st a_i\in A, b_i\in B, n \in\ZZ{>0}\} \), is an ideal.

\begin{solution} % 14.12
\end{solution}

\bd{Exercise 14.13.}
Find a positive integer \( a \) such that
\begin{enumerate}
  \item \( \cyc{a} = \cyc{3}\cyc{4} \)
  \item \( \cyc{a} = \cyc{6}\cyc{8} \)
  \item \( \cyc{a} = \cyc{m}\cyc{n} \)
\end{enumerate}

\begin{solution} % 14.3
\end{solution}

\bd{Exercise 14.14.}
Let \( A \) and \( B \) be ideals of a ring. Prove that \( AB\sseq A\cap B \).

\begin{solution} % 14.14
\end{solution}

\end{document}