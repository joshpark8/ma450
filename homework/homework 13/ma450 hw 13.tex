\documentclass{article}

\input{preamble}
\input{letterfont}
\input{macros}


% Misc helpers
\setlength{\parskip}{1em}
\setlength{\parindent}{0pt}
% \setlist[enumerate]{itemsep=-2mm}
\setlist[itemize]{itemsep=0mm}
\renewcommand{\arraystretch}{1.25} % space out table rows

% Title
\fancyhead[L]{Josh Park}
\fancyhead[C]{\tbo{MA 45000 Homework 13}}
\fancyhead[R]{Fall 2024}

% Document
\begin{document}
\tbo{Exercise 13.4.}
List all zero-divisors of \( \ZZ{20} \). Can you see a relationship between the zero-divisors of \( \ZZ{20} \) and the units of \( \ZZ{20} \)?

\begin{solution} % 13.4
  By definition zero-divisor, we wish to find all \( a_{\neq 0}\in R = \ZZ{20} \) such that \( \exists b_{\neq 0} \in R \) where \( ab\equiv 0 \pmod{20} \).
  That is, we wish to find all \( a_{\neq 0} \in R \) such that \( ab = 20n \) for some \( n\in\Z \) where \( b_{\neq 0}\in R \).
  % The prime factors of \( 20 = 2^2 \cdot 5 \) so we can say
  We can rewrite this as
  \begin{align*}
    ab = 20n \qimp \frac{ab}{20} = n.
  \end{align*}
  Suppose \( a \) is coprime to \( 20 \). Then by definition coprime, \( a \) and \( 20 \) share no common factors.
  So \( 2\ndivs a \) and \( 5\ndivs a \) which implies \( 2p+5q\ndivs a\ \forall p,q\in\Z \).
  That is, \( a \) is not divisible by any linear combination of 2 and 5 with integer coefficients, and consequently by any divisor (nor by any multiple) of \( 20 \). We know \( a \) is an integer, so \begin{align*}
    n = \frac{ab}{20} = a \cdot \frac{b}{20}\in\Z \qiff \frac{b}{20}\in\Z.
  \end{align*}
  Then \( b \equiv 0\pmod{20} \), but \( b\not\equiv 0 \) by definition \( b \) \contradiction. Thus \( a \) must not be coprime to \( 20 \).

  Suppose \( a \) is \emph{not} coprime to \( 20 \). Then by definition coprime, \( a \) shares at least one common factor with \( 20 \). Let this factor be \( p \). Then \( a = pq \) and \( 20 = pr \) for some \(q,r \in\ZZ{20} \). Suppose \( b = r \). Then, \begin{align*}
    ab = 20n \qiff pqr = prn \qiff r = n.
  \end{align*}
  We know \( r\in\Z \), so all numbers not coprime to 20 in \( \ZZ{20} \) are zero-divisors.

  So we have that \( a \) coprime to 20 \imp \( a \) not zero-divisor and \( a \) not coprime to 20 \imp \( a \) is zero-divisor. That is, \( a\in\ZZ{20} \) is a zero divisor \iff\ \( a \) is not coprime to \( 20 \). Thus the set of all zero divisors of \( \ZZ{20} \) is \{2, 4, 5, 6, 8, 10, 12, 14, 15, 16, 18\}.

  The set of zero-divisors of \( \ZZ{20} \) and the set of units of \( \ZZ{20} \) are disjoint and form a partition of \( \ZZ{20} \).
\end{solution}

\tbo{Exercise 13.24.}
Find a zero-divisor in \( \Zfi[i] = \{a+bi\mid a,b\in\Zfi\} \).

\begin{solution} % 13.24
  Let \( R = \Zfi[i] \). By definition zero-divisor, \( r_{\neq 0}\in R \) is a zero-divisor of \( R \) if there exists some \( s_{\neq 0}\in R \) such that \( rs \equiv 0 \pmod 5 \). Consider the elements \( r = 2+i \) and \( \bar{r} = 2-i \). Notice \begin{align*}
    rs = (2+i)(2-i) = 4 -2i + 2i + 1 = 5 + 0i \equiv 0 \pmod 5.
  \end{align*}
  Thus \( r \) is a zero-divisor of \( \Zfi[i] \).
\end{solution}

\tbo{Exercise 13.30.}
Let \( d \) be a positive integer. Prove that \( \![\sqrt d] = \{a + b\sqrt d\mid a,b\in\!\}\) is a field.

\begin{solution} % 13.30
  Viewed as an element of \( \bbR \), the multiplicative inverse of any element of the form \( a+b\sqrt d \) is \( 1/(a+b\sqrt d) \). To verify that \( \![\sqrt d] \) is a field, we must show \( 1/(a+b\sqrt d) \) can be written in the form \( \alpha+\beta\sqrt d \). Observe that
  \begin{align*}
    \frac{1}{a+b\sqrt d} = \frac{1}{a+b\sqrt d} \cdot \frac{a-b\sqrt d}{a-b\sqrt d} = \frac{a-b\sqrt{d}}{a\sq -ab\sqrt{d} + ab \sqrt{d} - b\sq d} = \frac{a}{a\sq-b\sq d} - \frac{b}{a\sq-b\sq d}\sqrt{d}.
  \end{align*}
  Thus \( \![\sqrt{d}] \) is a field.
\end{solution}

\tbo{Exercise 13.31.}
Let \( R \) be a ring with unity 1. If the product of any pair of nonzero elements of R is nonzero, prove that \( ab=1 \) implies \( ba=1 \).

\begin{solution} % 13.31
  We have that \( a_{\neq 0}, b_{\neq 0}\in R \imp ab \neq 0 \). Suppose \( ab = 1 \). Then \begin{align*}
    ab &= 1 \\
    aba &= a \\
    aba - a &= 0 \\
    a(ba - 1) &= 0
  \end{align*}
    Notice that \( a \) is nonzero, so \( ba - 1 = 0 \) and thus \( ba = 1 \).
\end{solution}

\tbo{Exercise 13.32.}
Let \( R = \{0, 2, 4, 6, 8\} \) under addition and multiplication modulo 10. Prove that \( R \) is a field.

\begin{solution} % 13.32
  By definition field, we need only verify each nonzero element of \( R \) has a multiplicative inverse. The nonzero elements of \( R \) are \{2, 4, 6, 8\}. By Exercise 12.2, we know the unity of \( R \) is 6. Thus, we must find some \( b\in R \) for each \( a\in\bbR \) such that \( ab = 6 \). Then, we can see that \begin{align*}
    &2\cdot 8 = 16 \equiv 6 \pmod{10}, &4\cdot 4 = 16 \equiv 6 \pmod{10}, \\
    &6\cdot 6 = 36 \equiv 6 \pmod{10}, &8\cdot 2 = 16 \equiv 6 \pmod{10}.
  \end{align*}
  Thus \( R \) is a field.
\end{solution}

\tbo{Exercise 13.42.}
Construct a multiplication table for \( \Ztw[i] \), the ring of Gaussian integers modulo 2. Is this ring a field? Is it an integral domain?

\begin{solution} % 13.42
  We know \( \Ztw[i] = \{a+bi\mid a,b\in\Ztw\} = \{0, i, 1, 1+i\} \)

  Then the multiplication table is
  \begin{center}
    \begin{tabular}{c | c c c c c}
                  & 0       & \( i \)   & 1         & \( 1+i \) \\ \cline{1-5}
        0         & 0       & 0         & 0         & 0  \\
        \( i \)   & 0       & 1         & \( i \)   & \( 1+i \)  \\
        1         & 0       & \( i \)   & 1         & \( 1+i \)  \\
        \( 1+i \) & 0       & \( 1+i \) & \( 1+i \) & 0
    \end{tabular}
  \end{center}
  Since \( (1+i)\sq = 0 \), it is a zero-divisor of \( \Ztw[i] \) by definition zero-divisor. Thus \( \Ztw[i] \) is not an integral domain by definition integral domain.
  Thus \( \Ztw[i] \) is not a field by definition field.
\end{solution}

\tbo{Exercise 13.43.}
The nonzero elements of \( \Zth[i] \) form an abelian group of order 8 under multiplication. Is it isomorphic to \( \Zei,\ \Zfo\edp\Ztw, \) or \( \Ztw\edp\Ztw\edp\Ztw \)?

\begin{solution} % 13.43
  We know \begin{align*}
    \Z_3[i] = \{a+bi \mid a,b\in\Zth\} = \{0, i, 2i, 1, 1+i, 1+2i, 2, 2+i, 2+2i\},
  \end{align*}
  so let \( G = \{i, 2i, 1, 1+i, 1+2i, 2, 2+i, 2+2i\} \).
  By thm, a group isomorphism must preserve the order of elements of the group.
  Thus we can test the orders of the elements of \( G \) to find an isomorphism.
  Consider the element \( \alpha=1+i \).
  Notice, \( (1+i)^2 = 2i \equiv -i \pmod 3 \), so \( (1+i)^4 = -1 \) and \( \order{\alpha} \) has order 8.
  By thm, the order of an element of an external direct product is the LCM of the orders of the elements.
  Then \( \Zfo\edp\Ztw \) and \( \Ztw\edp\Ztw\edp\Ztw \) can not have any elements of order 8, but \( \Zei \) can.
  Thus the set of nonzero elements of \( \Zth[i] \) is isomorphic to \( \Zei \).
\end{solution}

\begin{note}[Notation]
  I will use \( \srg \) to denote subring and \( \idl \) for ideal.
\end{note}

\tbo{Exercise 14.4.}
Find a subring of \( \Z\edp\Z \) that is not an ideal of \( \Z\edp\Z \).

\begin{solution} % 14.4
Consider the set \( R = \{(x,x)\mid x\in\Z\} \).
Notice \begin{align}
  (\alpha,\alpha) - (\beta,\beta) &= (\alpha-\beta, \alpha-\beta) \in R \label{srt c1}\\
  (\alpha,\alpha) \cdot (\beta,\beta) &= (\alpha\beta,\alpha\beta) \in R,
\end{align}
so \( R\sgp\Z\edp\Z \) by the subring test. Consider the elements \( a=(\alpha,\alpha)\in R \) and \( r = (\beta, \gamma) \in \Z\edp\Z \) such that \( \beta\neq\gamma \). Then, \begin{align*}
  ar = (\alpha,\alpha)\cdot (\beta,\gamma) = (\alpha\beta, \alpha\gamma).
\end{align*}
We know \( \beta\neq\gamma \), so \( \alpha\beta\neq\alpha\gamma \). Then \( (\alpha\beta,\alpha\gamma) \notin R \) whence \( R\notidl \Z\edp\Z \) by the ideal test.
\end{solution}

\tbo{Exercise 14.6.}
Find all maximal ideals in \\
\tbo{a.  \Zei \qquad b.  \ZZ{10} \qquad c.  \ZZ{12} \qquad d. }\Zn

\begin{solution} % 14.6
\begin{enumerate}[label=\tbo{\alph*.}]
  \item \Zei \\
    The proper ideals of \( \Zei \) are \( \pidl{0} = \tsrg \), \( \pidl{2} = \{0,2,4,6\}\), and \(\pidl{4} = \{0,4\}\).
    Since \( \pidl{0}\subset \pidl{4}\subset \pidl{2} \subset \Zei\), \( \pidl{2} \) is the only maximal ideal of \( \Zei \).
  \item \Zte \\
    The proper ideals of \( \Zte \) are \( \pidl{0} = \tsrg \), \( \pidl{2} = \{0,2,4,6,8\} \), and \( \pidl{5} = \{0,5\} \).
    Since \( \pidl{0}\subset \pidl{5}, \pidl{2} \subset \Zte \), \( \pidl{2} \) and \( \pidl{5} \) are the only maximal ideals of \( \Zte \).
  \item \ZZ{12} \\
    The proper ideals of \( \ZZ{12} \) are \( \pidl{0} = \tsrg \), \( \pidl{2} = \{0,2,4,6,8,10\} \), \( \pidl{3}=\{0,3,6,9\} \), \( \pidl{4}=\{0,4,8\} \), and \( \pidl{6}=\{0,6\} \).
    Notice \( \pidl{0}\subset\pidl{4} \subset \pidl{2} \subset \ZZ{12} \) and \( \pidl{0}\subset\pidl{6} \subset \pidl{3} \subset \ZZ{12} \). Thus \( \pidl{2} \) and \( \pidl{3} \) are the only maximal ideals of \( \ZZ{12} \).
  \item \Zn \\
    Suppose the prime factorization of \( n \) is \( n=\prod_{i=1}^{m} p_i^{k_i} \).
    I claim\( ^1 \) the only maximal ideals of \( \Zn \) are \( \pidl{p_1}, \pidl{p_2}, \ldots, \pidl{p_m} \), the principal ideals generated by the prime factors of \( n \).
    \begin{subproof}[Claim 1]
      First, I claim\( ^2 \) that the factor ring \( \fr{\Zn}{\pidl{d}} \) is isomorphic to \ZZ{n/d}.
      \begin{subproof}[Claim 2]
        By definition isomorphism, we must show there exists a bijective homomorphism \( \phi:\qg{\Zn}{\pidl{d}}\to \ZZ{n/d} \). Consider the mapping \( \phi:\fr{\Zn}{\pidl{d}}\to\ZZ{n/d} \) such that \(\phi(a+\pidl{d}) = a\mod{\frac{n}{d}} \).
        By definition homomorphism, we must show that \( \phi(a)+\phi(b) = \phi(a+b) \). Consider the elements \( a+\pidl{d}, b+\pidl{d}\in\fr{\Zn}{\pidl{d}} \). Then \begin{align*}
          \phi(a)+\phi(b) &= \left(a\mod{\frac{n}{d}}\right) + \left(b\mod{\frac{n}{d}}\right) \\
          &= (a+b)\mod{\frac{n}{d}} \\
          &= \phi(a+b).
        \end{align*}
        Thus \( \phi \) is a homomorphism. To show \( \phi \) is bijective, we must show \( \phi \) is both injective and surjective.

        To see that \( \phi \) is surjective, we must ensure every element in \( \ZZ{n/d} \) is mapped to by some element in \( \fr{\Zn}{\pidl{d}} \).
        Consider any \( c\in\ZZ{n/d} \).
        Then, let \( a\in\Zn \) such that \( a\mod d = a\mod{\frac{n}{d}} = c \).
        Then \( \phi(a+\pidl{d}) = a\mod{\frac{n}{d}} = c \).
        Thus \( \phi \) is surjective.

        To see that \( \phi \) is injective, suppose we have \( a+\pidl{d}, b+\pidl{d}\in\fr{\Zn}{\pidl{d}} \) such that \( \phi(a+\pidl{d}) = \phi(b+\pidl{d}) \).
        Then, \begin{align*}
          \phi(a+\pidl{d}) = \phi(b+\pidl{d}) &\iff a\mod{\frac{n}{d}} = b\mod{\frac{n}{d}} \iff a \equiv b\ \,\, \left(\text{mod }\frac{n}{d}\right) \iff a \equiv b\pmod d
        \end{align*}
        We know it is a property of cosets that \( aH=bH \iff a\in bH \).
        Thus, \begin{align*}
          a+\pidl{d} = b+\pidl{d} &\iff a\in b+\pidl{d} \iff a = b+dx,\ x\in\Zn \iff a\equiv b\pmod d
        \end{align*}
        Thus \( \phi(a+\pidl{d}) = \phi(b+\pidl{d})\iff a+\pidl{d}=b+\pidl{d} \) implies \( \phi \) is injective, whence \( \phi \) is an isomorphism from \( \fr{\Zn}{\pidl{d}} \) to \( \ZZ{n/d} \).
      \end{subproof}
      By Example 13.6 and Corollary 13.2.1, the ring \( \ZZ{n/d}\iso (\fr{\Zn}{\pidl{d}}) \) is a field \iff \( \frac{n}{d} \) is prime.
      By Theorem 14.4, the factor ring \( \fr{\Zn}{\pidl{d}} \) is a field \iff \( \pidl{d} \) is a maximal ideal of \( \Zn \).
      Thus, \( \pidl{d} \) is a maximal ideal of \( \Zn \) \iff \( \frac{n}{d} \) is prime.
    \end{subproof}
\end{enumerate}
\end{solution}
\pagebreak

\tbo{Exercise 14.10.}
If \( A \) and \( B \) are ideals of a ring, show that the \tit{sum} of \( A \) and \( B \), \( A+B = \{a+b\mid a\in A,b\in B\} \), is an ideal.

\begin{solution} % 14.10
  Let \( A \) and \( B \) be ideals of some ring \( R \). By definition ideal, \( ra,ar\in A \) for any \( a\in A,\ r\in R \). Similarly \( rb,br\in B \) for any \( b\in B,\ r\in R \). Consider the element \( a+b\in A+B \) and pick any \( r\in R \). Then \( r(a+b) = ra+rb \) and \( (a+b)r = ar+br \) by properties of multiplication. Since \( ra\in A \) and \( rb\in B \), \( ra+rb \in A+B \). Similarly, \( ar+br \in A+B \). Thus \( A+B \) is an ideal of \( R \) by def ideal.
\end{solution}

\tbo{Exercise 14.11.}
In the ring of integers, find a positive integer \( a \) such that
\begin{enumerate}[label=\tbo{\alph*.}]
  \item \( \pidl{a} = \pidl{2} + \pidl{3} \)
  \item \( \pidl{a} = \pidl{6} + \pidl{8} \)
  \item \( \pidl{a} = \pidl{m} + \pidl{n} \)
\end{enumerate}

\begin{solution} % 14.11
\begin{enumerate}[label=\tbo{\alph*.}]
  \item \( \pidl{a} = \pidl{2} + \pidl{3} \) \\
    By definition of principal ideal, \( \pidl{2}=\{2k\mid k\in\Z\} \) and \( \pidl{3}=\{3k\mid k\in\Z\} \).
    So any element \(\gamma\in \pidl{2} + \pidl{3} \) must be of the form \( \gamma=2\alpha+3\beta \), where \( \alpha,\beta\in\Z \).
    We know 2 and 3 are both prime and are thus relatively prime, so by Bezout's Identity there exist integers \( s,t \) such that \( 2s+3t=1 \). Notice that for any \( k\in\Z \), \begin{align*}
      k = k(2s+3t) = 2ks+3kt
    \end{align*} by properties of multiplication.
    Thus we can generate any integer \( k \) by letting \( \alpha = ks \) and \( \beta = kt \).
    Thus \( \pidl{2}+\pidl{3} = \Z = \pidl{1} \) and \( a=1 \).
  \item \( \pidl{a} = \pidl{6} + \pidl{8} \) \\
    By definition of principal ideal, \( \pidl{6}=\{6k\mid k\in\Z\} \) and \( \pidl{8}=\{8k\mid k\in\Z\} \).
    So any element \(\gamma\in\pidl{6} + \pidl{8} \) must be of the form \( \gamma=6\alpha+8\beta \), where \( \alpha,\beta\in\Z \). Notice that while 6 and 8 are not relatively prime, \begin{align*}
      \gamma=6\alpha+8\beta= 2(3\alpha+4\beta)
    \end{align*} where \( 3\alpha+4\beta \in \pidl{3}+\pidl{4} \).
    Let us momentarily switch our attention to finding some \( b \) such that \(\pidl{b} = \pidl{3}+\pidl{4} \).
    Since 3 and 4 are relatively prime, we can the same logic as in part a to find that \( \pidl b = \pidl 1 \).
    Thus, \( \pidl{6}+\pidl{8} = 2\Z = \pidl{2} \) and \( a=2 \).
  \item \( \pidl{a} = \pidl{m} + \pidl{n} \) \\
    By definition of principal ideal, \( \pidl{m}=\{mk\mid k\in\Z\} \) and \( \pidl{n}=\{nk\mid k\in\Z\} \).
    So any element \(\gamma\in\pidl{m} + \pidl{n} \) must be of the form \( \gamma=m\alpha+a\beta \), where \( \alpha,\beta\in\Z \).
    If \( m \) and \( n \) relatively prime, then we can follow the proof of part a and we are done.
    Suppose \( m \) and \( n \) are \tit{not} relatively prime. That is, \( \gcd(m,n) = d > 1\). So any element \( \gamma\in\pidl{m}+\pidl{n} \) must be of the form \( \gamma=m\alpha+n\beta \). Notice that \begin{align*}
      \gamma = m\alpha+n\beta = d\lt(\frac{m}{d}\alpha+\frac{n}{d}\beta\rt),
    \end{align*} where \( \frac{m}{d} \) and \( \frac{n}{d} \) have no common divisors by definition of GCD and are thus coprime.
    Thus by part a, \( \pidl{\frac{m}{d}}+\pidl{\frac{n}{d}}=\pidl{1}=\Z \).
    Thus \( \pidl{m}+\pidl{n} = d\Z = \pidl{d} \) and \( a = d = \gcd(m,n) \).
\end{enumerate}
\end{solution}

\tbo{Exercise 14.12.}
If \( A \) and \( B \) are ideals of a ring, show that the \tit{product} of \( A \) and \( B \), \( AB = \{a_1b_1 + a_2b_2 + \cdots + a_nb_n \mid a_i\in A, b_i\in B, n \in\ZZ{>0}\} \), is an ideal.

\begin{solution} % 14.12
  Let \( A \) and \( B \) be ideals of some ring \( R \).
  To show \( AB = \left\{\sum_{i=1}^n a_ib_i \mid a_i\in A, b_i\in B, n\in\ZZ{>0}\right\} \) is an ideal of \( R \), we use the ideal test.
  Suppose we have some \( x,y\in AB \).
  By def \( AB \), \begin{align*}
    x = \sum_{i=1}^{n} a_ib_i \qquad y = \sum_{j=1}^{m} a_j'b_j',
  \end{align*} where \( a_i,a_j'\in A \) and \( b_i,b_j'\in B \).
  Since \( x \) and \( y \) are arbitrary, let \( n<m \).
  Then, \begin{align*}
    x-y &= \sum_{i=1}^{n} a_ib_i - \sum_{j=1}^{m} a_j'b_j' \\
        &= a_1b_1 + a_2b_2 + \cdots + a_nb_n - a_1'b_1' - a_2'b_2' - \cdots - a_mb_m \\
        &= (a_1b_1-a_1'b_1') + (a_2b_2-a_2'b_2') + \cdots + (a_nb_n-a_nb_n) - a_{n+1}'b_{n+1}' - \cdots -a_m'b_m' \\
        &= \sum_{i=1}^n(a_i+a_i')(b_i-b_i') - \sum_{j=n+1}^{m} a_j'b_j'.
  \end{align*}
  Since \( A \) and \( B \) are ideals, they are closed under addition/subtraction and we can write \( a_i+a_i'=a_i''\in A \) and \( b_i-b_i' = b_i''\in B \).
  Then, \begin{align*}
    x-y &= \sum_{i=1}^{n}a_i''b_i'' - \sum_{j=n+1}^{m} a_j'b_j' \\
        &= \sum_{i=1}^{n}a_i''b_i'' + \sum_{j=n+1}^{m} (-a_j')b_j' \in AB.
  \end{align*}
  Thus \( AB \) is closed under subtraction.
  Let \( x\in AB \) and \( r\in R \). By definition, \begin{align*}
    x &= \sum_{i=1}^{n} a_ib_i,
  \end{align*} where \( a_i\in A \) and \( b_i\in B \).
  Since \( A \) and \( B \) are ideals, \( ra_i = \bar{a_i} \in A \) and \( b_i r = \bar{b_i} \in B \) for all \( r\in R \) by definition ideal.
  Then, \begin{align*}
    rx &= r\left(\sum_{i=1}^{n} a_ib_i\right) = \sum_{i=1}^{n} ra_ib_i = \sum_{i=1}^{n}\bar{a_i}b_i \in AB\\
    xr &= \left(\sum_{i=1}^{n} a_ib_i\right)r = \sum_{i=1}^{n} a_ib_ir = \sum_{i=1}^{n}a_i\bar{b_i} \in AB.
  \end{align*}
  So \( rx,xr\in AB \) for every \( x\in AB \) and every \( r\in R \). Thus \( AB \) is an ideal of \( R \) by the ideal test.
\end{solution}

\tbo{Exercise 14.13.}
Find a positive integer \( a \) such that
\begin{enumerate}[label=\tbo{\alph*.}]
  \item \( \pidl{a} = \pidl{3}\pidl{4} \)
  \item \( \pidl{a} = \pidl{6}\pidl{8} \)
  \item \( \pidl{a} = \pidl{m}\pidl{n} \)
\end{enumerate}

\begin{solution} % 14.3
\begin{enumerate}[label=\tbo{\alph*.}]
  \item \( \pidl{a} = \pidl{3}\pidl{4} \)\\
    Elements of \( \pidl{3}\pidl{4} \) are of the form \begin{align*}
      \sum_{i=1}^{n}3\alpha_i4\beta_i = \sum_{i=1}^{n}12\alpha_i\beta_i = 12\sum_{i=1}^{n}\alpha_i\beta_i = 12s \in \pidl{12},
    \end{align*}
    where \( \alpha_i,\beta_i\in \Z \) and \( s=\sum_{i=1}^{n}\alpha_i\beta_i \).
    So, \( \pidl{3}\pidl{4} \sseq \pidl{12} \).
    Also since \( 12\in \pidl{3}\pidl{4} \), we have that \( \pidl{12}\sseq \pidl{3}\pidl{4} \).
    Thus \( \pidl{3}\pidl{4} = \pidl{12} \) and \( a=12 \).
  \item \( \pidl{a} = \pidl{6}\pidl{8} \)\\
    Elements of \( \pidl{6}\pidl{8} \) are of the form \begin{align*}
      \sum_{i=1}^{n}6\alpha_i8\beta_i = \sum_{i=1}^{n}48\alpha_i\beta_i = 48\sum_{i=1}^{n}\alpha_i\beta_i = 48s \in \pidl{48},
    \end{align*} where \( \alpha_i,\beta_i\in \Z \) and \( s=\sum_{i=1}^{n}\alpha_i\beta_i \).
    So, \( \pidl{6}\pidl{8} \sseq \pidl{48} \).
    Also since \( 48\in \pidl{6}\pidl{8} \), we have that \( \pidl{48}\sseq \pidl{6}\pidl{8} \).
    Thus \( \pidl{6}\pidl{8} = \pidl{48} \) and \( a=48 \).
  \item \( \pidl{a} = \pidl{m}\pidl{n} \)\\
    Elements of \( \pidl{m}\pidl{n} \) are of the form \begin{align*}
      \sum_{i=1}^{n}m\alpha_in\beta_i = \sum_{i=1}^{n}mn\alpha_i\beta_i = mn\sum_{i=1}^{n}\alpha_i\beta_i = mn s \in \pidl{mn},
    \end{align*} where \( \alpha_i,\beta_i\in \Z \) and \( s=\sum_{i=1}^{n}\alpha_i\beta_i \).
    So, \( \pidl{m}\pidl{n}\sseq \pidl{mn} \).
    Also since \( mn\in \pidl{m}\pidl{n} \), we have that \( \pidl{mn}\sseq\pidl{m}\pidl{n} \).
    Thus \( \pidl{m}\pidl{n} = \pidl{mn} \) and \( a=mn \).
\end{enumerate}
\end{solution}

\tbo{Exercise 14.14.}
Let \( A \) and \( B \) be ideals of a ring. Prove that \( AB\sseq A\cap B \).

\begin{solution} % 14.14
  Let \( A \) and \( B \) be ideals of a ring \( R \).
  Let \( x\in AB \).
  By definition of \( AB \), we can write \begin{align*}
    x = \sum_{i=1}^{n}a_ib_i,
  \end{align*} where \( a_i\in A \) and \( b_i\in B \). We wish to show \( x\in A \) and \( x\in B \).
  Since \( A \) and \( B \) are ideals of \( R \), \( \alpha\in R \) for all \( \alpha\in A \) and likewise for \( B \).
  Then each term \( a_ib_i \) can be written \( a_ir \) or \( rb_i \) where \( r\in R \).
  By def ideal, \( a_ir\in A \) and \( rb_i\in B \).
  Also by def ideal, \( A \) and \( B \) are closed under addition.
  Thus \( x\in A \) and \( x\in B \) and \( x\in A\cap B \).
  Thus \( AB\sseq A\cap B \).
\end{solution}

\end{document}