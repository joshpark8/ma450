\documentclass{article}

% Packages
% \usepackage[margin=1in]{geometry}
% \usepackage{amsfonts}
% \usepackage{amsmath}
% \usepackage{amssymb}
% \usepackage{amsthm}
% \usepackage{CJK}
% \usepackage{enumitem}
% \usepackage{epsf}
% \usepackage{etoolbox}
% \usepackage{tcolorbox}
% \usepackage{float}
% \usepackage{graphicx}
% \usepackage{latexsym}
% \usepackage{systeme}

\input{mypreamble.tex}
\input{letterfont.tex}
\input{macros.tex}


% Misc helpers
\setlength{\parskip}{1em}
\setlength{\parindent}{0pt}
\setlist[enumerate]{itemsep=0mm}
\setlist[itemize]{itemsep=0mm}

% Title
\author{Josh Park}
\date{\vspace*{-1em}Fall 2024}
\title{\vspace*{-2em}MA 450 Homework 10\vspace*{-1em}}

% Document
\begin{document}
\maketitle
\textbf{Exercise 10.31} Suppose that \(\phi\) is a \homo\ from \(U(30)\) to \(U(30)\) and that \(\kerphi = \{1,11\}\). If \(\phi(7) = 7\), find all elements of \(U(30)\) that map to 7.
\newpage

\textbf{Exercise 10.32} Find a homomorphism \(\phi\) from $U(30)$ to $U(30)$ with kernel {1, 11} and $\phi$(7) 5 7.
\newpage

\textbf{Exercise 10.41} (Second Isomorphism Theorem) If $K$ is a subgroup of $G$ and $N$ is a normal subgroup of $G$, prove that $K/(K > N)$ is isomorphic to $KN/N$.
\newpage

\textbf{Exercise 10.42} (Third Isomorphism Theorem) If $M$ and $N$ are normal subgroups of $G$ and $N \sgp M$, prove that $(G/N)/(M/N) \cong G/M$.
\newpage

\textbf{Exercise 11.4} Calculate the number of elements of order 2 in each of $\Z_{16},\ \Z_{8} \edp \Z_{2},\ \Z_{4} \edp \Z_{4},$ and $\Z_{4} \edp \Z_{2} \edp \Z_{2}$. Do the same for the elements of order 4.
\newpage
s
\textbf{Exercise 11.8} Show that there are two Abelian groups of order 108 that have exactly 13 subgroups of order 3.
\newpage

\textbf{Exercise 11.15} How many Abelian groups (up to isomorphism) are there \vspace{-15pt}
\newpage

\begin{enumerate}[label=\textbf{\alph*.}]
    \item of order 6?
    \item of order 15?
    \item of order 42?
    \item of order \(pq\), where \(p\) and \(q\) are distinct primes?
    \item of order $pqr$, where $p$, $q$, and $r$ are distinct primes?
    \item Generalize parts d and e.
\end{enumerate}

\textbf{Exercise 11.26} Let $G = \{1, 7, 17, 23, 49, 55, 65, 71\}$ under multiplication modulo 96. Express $G$ as an external and an internal direct product of cyclic groups.
\newpage

\textbf{Exercise 11.28} The set $G = \{1, 4, 11, 14, 16, 19, 26, 29, 31, 34, 41, 44\}$ is a group under multiplication modulo 45. Write $G$ as an external and an internal direct product of cyclic groups of prime-power order.
\newpage

\textbf{Exercise 11.30} Suppose that $G$ is an Abelian group of order 16, and in computing the orders of its elements, you come across an element of order 8 and two elements of order 2. Explain why no further computations are needed to determine the isomorphism class of $G$.
\newpage

\textbf{Exercise 11.36} Suppose that $G$ is a finite Abelian group. Prove that $G$ has order $p^n$, where $p$ is prime, if and only if the order of every element of $G$ is a power of $p$.
\newpage

\end{document}